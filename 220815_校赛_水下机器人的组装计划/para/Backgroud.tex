近年来,随着社会经济的发展和科学技术的进步,管道越来越多的运用在我们的生活当中。
管道运输是最实用、最经济的运输方式,所以管道运输在生活和生产中的使用也越来越广泛\cite{智能管道清理机器人}。
其中,自来水管道已经进入千家万户。
当净水厂的出厂水经过供水管网,输送给用户时,其在供水管网中将会发生复杂的生物、物理、化学反应\cite{基于升力法的贯流式水轮机叶片设计及可行性分析}。
导致污垢在自来水管道壁上积累,影响自来水的品质和用途。
自来水管道清理机器人是一种体型较小,使用机械臂以辅助完成管道清理任务的自动装置。
相比与传统人工清理方式,机器人清理具有及时、高效的优点,因此倍受水务公司和住户的青睐。

\section{问题重述与分析}
某工厂生产的WPCR装置由3个容器艇(用A表示)、4个机器臂(用B表示)、5个动力系统(用C表示)组装而成。
而A、B、C由以下部件组成:
\begin{itemize}
\item 容器艇(A)由6个控制器(A1)、8个划桨(A2)和2个感知器(A3)组成,组装需消耗3个工时;
\item 机器臂(B)由2个力臂组件(B1)和4个遥感器(B2)组成,组装需消耗5个工时;
\item 动力系统(C)由8个蓄电池(C1)、2个微型发电机(C2)和12个发电螺旋(C3)组成,组装需消耗5个工时。
\end{itemize}

工厂在某一天生产组件产品时,都要付出一个与生产数量无关的固定成本,称为生产准备费用。
而当一天结束时仍有某部件的库存,则须付出额外的库存费用。
每次生产计划的计划期为一周,提供的最终产品为WPCR,以满足订单需要,不可轻易缺货断供。

%需修改
为了最大化经济效益,帮助生产工厂做出决策,本文建立了???模型。

\subsection{问题一}
题目要求生产周期开始时没有任何组件库存,周期结束后也不留下任何组件库存。
在部件采购与WPCR组装无延迟的基础上,要求总成本最小。
因此可直接以总成本为目标函数建立线性规划模型,在题目所给约束条件下求得最优解,以制定每周7天的生产计划。


\subsection{问题二}
题目要求在问题一模型的基础上,考虑组件入库延迟对模型和生产计划的影响。
与问题一组件入库无延迟不同,问题二中组装产品所需的组件要提前一天入库才能参与生产。
因此,本文在问题一模型约束条件中,添加了令工厂只得使用前一天的组件库存生产新组件的限制。
通过求解新模型,得到记入组件入库延迟的最优7天的生产计划。

\subsection{问题三}



\subsection{问题四}
题目中要求在供应商和转运商有限的情况下,评估企业每周产能的上限。
本题取消生产企业每周产能上限的同时,也消除了原材料的库存问题。
而通过分析供货数据不难发现,向企业输送的最大供给量主要取决于转运商的转运能力。
本问题也转化为在转运商以最大能力运输原材料时,一家供应商每周供应的原材料尽量由一家转运商运输。
因此,本文主要对转运商的转运方案进行优化,并据此给出未来24周的订购和转运方案。

最后我们对模型进行了中肯的评价和适当的推广。

\section{符号说明} % (fold)
\label{sec:符号说明}

\begin{table}[!htbp]
\centering
\caption[文中符号所用说明]{文中符号所用说明}
\resizebox{\linewidth}{!}{
\begin{tabular}{cl}
\toprule[1.5pt]
符号 & 说明 \\
\midrule[1pt]
$x_t^r$ & 第$t$周,组件$r$(包括WPCR)的组装数量. \\
$y_t^r$ & 第$t$周,组件$r$(包括WPCR)的库存数量. \\
$d_t$ & 第$t$周,WPCR的外部需求数量. \\
$M_t$ & 第$t$周,组件$r$(仅包括A、B和C)的生产总工时限制. \\
$s^r$ & 组件$r$(包括WPCR)的生产准备费用. \\
$h^r$ & 组件$r$(包括WPCR)的单件库存费用. \\
$c^r$ & 组件$r$(仅包括A、B和C)的单件工时消耗. \\
\bottomrule[1.5pt]
\end{tabular}
}
\end{table}

% section 符号说明 (end)