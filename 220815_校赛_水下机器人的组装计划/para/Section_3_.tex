\section{检修方案设计} % (fold)
\label{sec:检修方案设计}

为了保障生产安全、提高工厂生产条件,生产工厂常常有计划地对关键设备进行检修升级。
对关键设备的检修势必会影响其正常生产,因此有必要针对工厂具体生产条件,合理设计检修安排,保证工厂安全、高效运转。

\subsection{模型的建立与求解} % (fold)
\label{sub:模型的建立与求解}

题目要求制定生产工厂停工检修计划,并使总成本最小。
为了表示当天$t$是否检修,本文引入了0$\textendash$1变量$\mu_t$,等于1表示当天进行检修,等于0表示不进行检修。

\begin{equation}
	\mu_{t} \in\{0,1\}
\end{equation}

设置7次停工检修,每次检修时间为1天。

检修之后关键设备生产能力会略有上升(10\%),并以2\%/天的速率衰减到0。

检修日无法生产,且意两次检修之间要相隔6天以上。
% subsection 模型的建立与求解 (end)
% section 检修方案设计 (end)