

 \maketitle
 %\tableofcontents
 \begin{abstract}

近年来,随着生活条件的改善,自来水管道逐渐走进千家万户。
然而,自来水管道容易在长期使用中积累污垢,对供水品质有较大影响。
本文立足于自来水管道清理机器人工厂生产条件,逐步改进其材料订购和生产计划,以降低企业生产成本,并为相关企业提供参考。

对于问题一,我们以总成本最低为目标,建立了符合工厂生产特点的线性规划模型,并制定出工厂一周最优生产计划。
首先以工厂生产成本构建目标函数,并进一步划分为生产准备费用和存贮费用,以分别进行优化。
在题中采购和组装无延迟的前提下,为了模拟工厂生产条件,提出了一系列约束条件:库存数量约束、WPCR需求约束、工时约束等。
最后利用数学工具对模型进行求解,制定工厂的7天生产计划,并结合图表进行评估。

对于问题二,针对题目中提出的组件不可直接投入生产的限制,本文在问题一的模型基础上进行改进,得到了工厂所需的优化生产计划。
由于组装产品所需的组件要提前一天入库才能参与生产,为了保障工厂每天的正常生产,添加了令工厂只得使用前一天的组件库存生产新组件的新约束。
通过求解改进过的线性规划模型,得到最优7天生产计划,结合图表进行评估。

对于问题三,为了保障生产的持续性,工厂将在210天内设置7个检修日。
据此,本文考虑如何安排生产工厂停工检修,可令总成本最小。
在上文模型的基础上,为了表示当天是否检修,而引入逻辑0$\textendash$1变量$\mu_t$参与线性约束。
通过求解线性规划模型,得到了检修日安排的最优方案。

对于问题四,工厂外部订单数量不再已知,需要根据历史订单数据结合上文模型制定未来生产方案。
因此,建立ARIMA$(p, T , q)$模型,以预测未来7天订单数。
通过进行显著性水平$\alpha$来计算正常支付的概率,将通过检验的预测数据作为需求数代入问题2的优化模型求解,得出未来7天的生产计划。

最后我们对模型进行了中肯的评价和适当的推广。
\keywords{0$\textendash$1规划\quad   ARIMA模型\quad 线性规划}
\end{abstract}
