

 \maketitle
 %\tableofcontents
 \begin{abstract}

近年来,随着生活条件的改善,自来水管道逐渐走进千家万户.
然而,自来水管道容易在长期使用中积累污垢,对供水品质有较大影响.
本文立足于自来水管道清理机器人工厂生产条件,逐步改进其材料订购和生产计划,以降低企业生产成本,并为相关企业提供参考.

对于问题一,我们以总成本最低为目标,建立了符合工厂生产特点的线性规划模型,并制定出工厂一周最优生产计划.
首先以工厂生产成本构建目标函数,并引入0$\textendash$1变量表示并计算生产准备费.
在题中采购和组装无延迟的前提下,为了模拟工厂生产条件,提出了一系列约束条件:库存数量约束、WPCR需求约束、工时约束等.
最后利用GUROBI对模型进行求解,制定工厂的7天生产计划,并进行了可视化结合库存数量变化情况认为工厂频繁开工节约仓储费用.

对于问题二,针对题目中提出的组件不可直接投入生产的限制,本文在问题一的模型基础上进行改进,得到了工厂所需的优化生产计划.
由于组装产品所需的组件要提前一天入库才能参与生产,为了保障工厂每天的正常生产,添加了令工厂只得使用前一天的组件库存生产新组件的新约束.
通过求解改进过的线性规划模型,得到最优7天生产计划,结合图表进行评估,发现工厂前期生产量较大、但后期较小.

对于问题三,为了保障生产的持续性,工厂将在210天内设置7个检修日.
据此,本文考虑如何安排生产工厂停工检修,可令总成本最小.
在上文模型的基础上,为了表示当天是否检修,而引入逻辑0$\textendash$1变量参与线性约束.
通过求解线性规划模型,得到了局部最优解作为检修日安排的最优方案.

对于问题四,工厂外部订单数量不再已知,需要根据历史订单数据结合上文模型制定未来生产方案.
因此,建立ARIMA$(p, T , q)$模型,以预测未来7天订单数.
通过进行显著性水平$\alpha$来计算正常支付的概率,将通过检验的预测数据作为需求数代入问题2的优化模型求解,得出未来7天的生产计划.

最后我们对模型进行了中肯的评价和适当的推广.
\keywords{0$\textendash$1规划\quad   ARIMA模型\quad 线性规划}
\end{abstract}
