\printbibliography
\newpage
%附录
\begin{appendices}

\section{源程序}
\subsection{ELOL.py} % (fold)
\label{sub:elol}

% subsection elol (end)
\begin{lstlisting}[language=python]
# %%
#引入序列长度
from dataclasses import dataclass
import numpy as np
import pandas as pd
import torch
from PyEMD import EMD
import matplotlib.pyplot as plt
from IPython.display import clear_output
import pickle
import os


#时间序列类
#传入数据,返回一个指定长度的
class DataSeq:
    def __init__(self, dataSet:np.array, step:int):
        self.data = dataSet
        self.step = step
        self.len = len(self.data) - self.step + 1

    def __getitem__(self,index,step=None):
        if step == None:
            step = self.step
        if isinstance(index,slice):
            return self.getkeys(index,step)
        return self.getkey(index,step)

    def getkey(self,index,step):
        data = self.data[index : index + step]
        assert len(data) == self.step, f'detaData out of index! length is {self.len} but index is {index}'
        return data

    def getkeys(self,indexSlice,step):
        start,stop = indexSlice.start, indexSlice.stop
        if start == None :
            start = 0
        if stop == None:
            stop = self.len
        else:
            stop = stop - 1
        ls = []
        for index in range(start,stop+1):
            ls.append(self.getkey(index,step))
        datas = np.array(ls)
        return datas

    def __len__(self):
        return self.len

    def __str__(self):
        return str(self.data)



class DateData(torch.utils.data.Dataset):

    def __init__(self,dataSet:np.array,length = 256, imf_num = 4-1, pre_num=1, dataSetWindows=1000):
        super().__init__()

        self.rawData = dataSet.copy()
        self.rawLen = len(self.rawData)
        step = min(self.rawLen, dataSetWindows) - length
        self.step = step
        dataSet = dataSet[-dataSetWindows:]

        #获取 imf_num数
        self.X = DataEMD(dataSet[:-pre_num],length,step,imf_num)
        self.imf_num = min(self.X.imf_num - 1, imf_num)
        self.Data = DataEMD(dataSet,length,step,self.imf_num)

        self.len = len(self.Data) - 1
        self.params = [length , self.imf_num, pre_num, dataSetWindows]

    def __getitem__(self,index):
        if isinstance(index,slice):
            assert index.stop is None or len(self.X) >= index.stop, 'detaData out of index!'
        else:
            assert self.len >= index, f'detaData out of index! length is {self.len} but index is {index}'
        return self.Data[index].astype(np.float32), self.Data[index+1].astype(np.float32)

    def __len__(self):
        return self.len

    def update(self,newData):
        dataSet = np.concatenate([self.rawData,newData],axis=0)
        self.__init__(dataSet,*self.params)

    def copy():
        pass

# %%
class DataEMD(DataSeq):
    def __init__(self, dataSet:np.array,length:int, step:int,imf_num=-1,emd = EMD()):
        self.rawData = dataSet
        data = emd(self.rawData,max_imf=imf_num).transpose(1,0)
        all_length,self.imf_num = data.shape
        super().__init__(DataSeq(data,length), step)
        self.emd = emd

    def update(self,data):
        pass


# %%
def re_EMD(data):
    batch, step, length, imf_num = data.shape
    ls = []
    for i in range(imf_num):
        chose = Chose_Y(i)
        temp_y = get_num(chose(data))
        ls.append(temp_y)
    d = np.concatenate(ls,axis=1)

    return d.sum(axis=1)


# %%
def get_num(Y):
    return Y.cpu().detach().numpy()

# %%
class Chose(torch.nn.Module):
    def __init__(self, imf):
        super().__init__()
        self.imf = imf

    def forward(self,X):
        #input = b * s * l * imf
        y = X[:,:,:,self.imf]
        O = y.transpose(1,0)
        return O

class Chose_Y(torch.nn.Module):
    def __init__(self, imf,length=1):
        super().__init__()
        self.imf = imf
        self.length = length

    def forward(self,X):
        #input = b * s * l * imf
        b,s,l,imf = X.shape
        y = X[:,:,-self.length:,self.imf]
        O = y.reshape(b*s,-1)
        return O


class MyLstm_reg(torch.nn.Module):
    def __init__(self,length,hidden, layer=2,out_num = 1):
        super().__init__()
        self.LSTM = torch.nn.LSTM(length,hidden,num_layers=layer)
        self.state = None
        self.linear = torch.nn.Linear(hidden,out_num)

    def forward(self,X):

        y, self.states= self.LSTM(X)

        s,b,l = y.shape
        h = y.reshape(s*b,l)
        o = self.linear(h)
        return o


class add_net(torch.nn.Module):
    def __init__(self,axis = 2):
        super().__init__()
        self.axis = axis

    def forward(self,X):
        """
        X = b * s * imf * hidden
        """
        return X.sum(axis = self.axis)

# %%
class Trans:

    # 标准化类,默认使用正态标准化
    def __init__(self, trans_fn = None, re_trans_fn = None):
        self.re_state = False
        if trans_fn == None:
            self.trans_fn = self._stand
            self.re_trans_fn = self._re_stand
        elif re_trans_fn != None:
            self.trans_fn = trans_fn
            self.re_trans_fn = re_trans_fn
        else:
            RuntimeError('没有传入恢复函数!')

    def _stand(self,data):
        if self.re_state == False:
            self.re_trans_params = [data.mean(),data.std()]
            self.re_state = True
        new_data = (data - self.re_trans_params[0]) / (self.re_trans_params[1])
        return new_data
    def _re_stand(self,data):
        temp_data = data * (self.re_trans_params[1])
        re_data = temp_data + self.re_trans_params[0]
        return re_data

    def _max_min(self, data):
        if self.re_state == False:
            self.re_trans_params = [data.min(),data.max()]
            self.re_state = True
        new_data = (data - self.re_trans_params[0]) / (self.re_trans_params[1] - self.re_trans_params[0])
        return new_data

    def _re_max_min(self,data):
        temp_data = data * (self.re_trans_params[1] - self.re_trans_params[0])
        re_data = temp_data + self.re_trans_params[0]
        return re_data

# %%
def grad_clipping(net, theta):  #@save
    """裁剪梯度"""
    if isinstance(net, torch.nn.Module):
        params = [p for p in net.parameters() if p.requires_grad]
    else:
        params = net.params
    norm = torch.sqrt(sum(torch.sum((p.grad ** 2)) for p in params))
    if norm > theta:
        for param in params:
            param.grad[:] *= theta / norm

# %%
def train_begin(net, data_iter, epoch, imf, device, lr=0.01,opim_fn = torch.optim.Adam,out_num = 1000, show_pic = True):
    # 初始化网络,在初次拟合训练时使用
    ls = []
    #迁移至GPU
    net.to(device)
    #初始 chose_Y 并选择 imf
    chose = Chose_Y(imf)
    opimter = opim_fn(net.parameters(),lr)

    loss = torch.nn.MSELoss()
    X,Y = next(iter(data_iter))

    trans = Trans()
    y = chose(Y)
    y_new = trans.trans_fn(y)
    out_time = torch.log(y_new.std() * y_new.abs().mean()/(500) + 1)

    for i in range(epoch):
        one_temp = []
        for X,Y in data_iter:
            X_new = trans.trans_fn(X)
            Y_new = trans.trans_fn(Y)
            yhat = net(X_new.to(device))
            y1 = chose(Y_new.to(device))
            l = loss(yhat[5:],y1[5:])
            if i> out_num and get_num(out_time) > get_num(l):
                return ls,trans
            opimter.zero_grad()
            l.backward()
            grad_clipping(net,1)
            opimter.step()
            one_temp.append(l)
            #ls.append(l)
        one_temp = torch.stack(one_temp,dim=0)
        ls.append(one_temp.max())

        if i % 20 == 0 and show_pic == True:
            clear_output(wait=True)
            print('out_time is loss less than',out_time,'and i is',i)
            plt.cla()
            temp = torch.stack(ls,dim=0)
            plt.plot(get_num(temp)[-100:])
            plt.show()

            plt.plot(get_num(y1)[:],'r')
            plt.plot(get_num(yhat)[:],'b',alpha=0.4)
            plt.show()
    return ls,trans

# %%
def get_net(imf,length = 64 ,hidden=256):
    #返回一个网络
    net = torch.nn.Sequential(
    Chose(imf),
    MyLstm_reg(length,hidden=hidden)
    )
    return net

# %%
def load_nets_and_trans(nets_path_ls, trans_path,length=64,hidden=256):
    nets = []
    imf_num = len(nets_path_ls)
    with open(trans_path,'rb') as f:
        trans_ls = pickle.load(f)
    for i in range(imf_num):
            net = get_net(i,length=length,hidden=hidden)
            net.load_state_dict(torch.load(nets_path_ls[i]))
            nets.append(net)
    return nets, trans_ls

# %%
def train_all_net(data_iter, imf_num, device, lr = 0.0005, min_epoch = 800, max_epoch = 10000, root_path='net',net_suffix='_lstm.pkl',trans_suffix= 'trans_ls.info',
                opim_fn = torch.optim.Adam,show_pic = False, length=64, hidden=256):
    nets= []
    trans_ls = []
    for i in range(imf_num):
        temp_net = get_net(i,length, hidden)
        ls, trans_one = train_begin(temp_net, data_iter, max_epoch,
                                    i,device, lr = lr, out_num = min_epoch,opim_fn=opim_fn,
                                    show_pic = show_pic)
        trans_ls.append(trans_one)
        torch.save(temp_net.state_dict(),root_path+ f'\\{i}'+net_suffix)
        net = get_net(i,length, hidden)
        net.load_state_dict(torch.load(root_path+ f'\\{i}'+net_suffix))
        nets.append(net)
    dump = pickle.dumps(trans_ls)
    with open(root_path+ '\\'+trans_suffix,'wb') as f:
        f.write(dump)
    return nets, trans_ls


# %%
def update_net(data_iter, nets, imf_num, device, lr = 0.0005, min_epoch = 800, max_epoch = 10000, root_path='net',net_suffix='_lstm.pkl',trans_suffix= 'trans_ls.info',
                opim_fn = torch.optim.Adam, show_pic = False, dump_local = False, length=64, hidden=256):
    for net in nets:
        net.train()
    trans_ls = []
    for i in range(imf_num):
        ls, trans_one = train_begin(nets[i], data_iter, max_epoch, i,device,
                                    lr =  lr,out_num = min_epoch,opim_fn=opim_fn, show_pic= show_pic)
        trans_ls.append(trans_one)
        if dump_local == True:
            torch.save(nets[i].state_dict(),root_path+ f'\\{i}'+net_suffix)
            net = get_net(i, length, hidden)
            net.load_state_dict(torch.load(root_path+ f'\\{i}'+net_suffix))
            nets[i] = net

    if dump_local == True:
        dump = pickle.dumps(trans_ls)
        with open(root_path+ '\\'+trans_suffix,'wb') as f:
            f.write(dump)
    return nets, trans_ls

# %%
def predict_one(nets,trans_ls,data_iter):
    for net in nets:
        net.eval()
        net.to('cpu')
    X,Y = next(iter(data_iter))
    new_in = torch.cat([X,Y[0:1,-2:-1]],dim=1)
    imf_num = len(nets)

    pred_Y = torch.zeros_like(nets[0](new_in))

    for i in range(imf_num):
        trans_in_i = trans_ls[i].trans_fn(new_in)
        trans_Y = nets[i](trans_in_i)
        pred_Y += trans_ls[i].re_trans_fn(trans_Y)
    return pred_Y



# %%
class ELOL:
    """
    Emd LSTM OnLine Learning Module
    """
    def __init__(self,length,imf_num,hidden,rawData, device, pre_num=1, dataSetWindows=1000):

        self.length = length

        self.hidden = hidden
        self.data = DateData(rawData, length, imf_num-1, pre_num=pre_num, dataSetWindows = dataSetWindows)
        self.imf_num = self.data.imf_num
        self.data_iter = torch.utils.data.DataLoader(self.data, batch_size = 1)
        self.device = device

    def init_nets(self, lr = 0.0005, min_epoch = 2000,
                    max_epoch = 10000, root_path='net',
                    net_suffix='_lstm.pkl',trans_suffix= 'trans_ls.info',
                    opim_fn = torch.optim.Adam,show_pic = True):

        try:
            os.mkdir(root_path)
        except:
            print(f'文件夹 {root_path} 已经存在……开始训练网络')
        self.nets, self.trans_ls = train_all_net(self.data_iter,self.imf_num + 1, self.device,
                                                lr = lr, min_epoch = min_epoch,
                                                max_epoch = max_epoch, root_path = root_path,
                                                net_suffix = net_suffix,trans_suffix= trans_suffix,
                                                opim_fn = opim_fn,show_pic = show_pic, length=self.length, hidden = self.hidden)


    def load_nets_and_trans(self, nets_path, trans_path):
            self.nets, self.trans_ls = load_nets_and_trans(nets_path,trans_path,
                                                            length=self.length, hidden= self.hidden)


    def update_data_and_net(self,data , lr = 0.005,
                        min_epoch = 500, max_epoch = 10000,
                        root_path='net',net_suffix='_lstm.pkl',
                        trans_suffix= 'trans_ls.info',
                        opim_fn = torch.optim.Adam,
                        show_pic = False, dump_local = False):
        self.data.update(data)
        self.data_iter = torch.utils.data.DataLoader(self.data, batch_size = 1)
        self.nets, self.trans_ls = update_net(self.data_iter, self.nets, self.imf_num+1, self.device,
                                            lr=lr, min_epoch= min_epoch, max_epoch=max_epoch,
                                            root_path=root_path, net_suffix=net_suffix, trans_suffix= trans_suffix,
                                            opim_fn=opim_fn, show_pic= show_pic, dump_local=dump_local,length=self.length, hidden=self.hidden)


    def predict(self):
        self.pred_Y =  predict_one(self.nets,self.trans_ls,self.data_iter)
        return get_num(self.pred_Y[-1])


\end{lstlisting}

\subsection{header.py}

\begin{lstlisting}[language=python]
from ELOL import MyLstm_reg,pd,torch,Trans,get_num,grad_clipping,clear_output,plt,np
import pickle

def pre_ABC(x,data_index = 2):
    if x['材料分类'] == "A":
        x[data_index:] =  x[data_index:] / 0.6
    if x['材料分类'] == "B":
        x[data_index:] =  x[data_index:] / 0.66
    if x['材料分类'] == "C":
        x[data_index:] =  x[data_index:] / 0.72
    return x

def re_pre_ABC(x,data_index = 2):
    if x['材料分类'] == "A":
        x[data_index:] =  x[data_index:] * 0.6
    if x['材料分类'] == "B":
        x[data_index:] =  x[data_index:] * 0.66
    if x['材料分类'] == "C":
        x[data_index:] =  x[data_index:] * 0.72
    return x

# %%
class DataSeq:
    def __init__(self, dataSet:np.array, step:int):
        self.data = dataSet
        self.step = step
        self.len = len(self.data) - self.step + 1

    def __getitem__(self,index,step=None):
        if step == None:
            step = self.step
        if isinstance(index,slice):
            return self.getkeys(index,step)
        return self.getkey(index,step)

    def getkey(self,index,step):
        data = self.data[index : index + step]
        assert len(data) == self.step, f'detaData out of index! length is {self.len} but index is {index}'
        return data

    def getkeys(self,indexSlice,step):
        start,stop = indexSlice.start, indexSlice.stop
        if start == None :
            start = 0
        if stop == None:
            stop = self.len
        else:
            stop = stop - 1
        ls = []
        for index in range(start,stop+1):
            ls.append(self.getkey(index,step))
        datas = np.array(ls)
        return datas

    def __len__(self):
        return self.len

    def __str__(self):
        return str(self.data)

# %%
class PureData(torch.utils.data.Dataset):
    def __init__(self,X,Y):
        super().__init__()
        self.X = X
        self.Y = Y
        self.len = len(X)

    def __getitem__(self,index,step=None):
        return self.X[index],self.Y[index]
    def __len__(self):
        return self.len


# %%
def train_begin(net, data_iter, epoch, device, lr=0.01,opim_fn = torch.optim.Adam,out_num = 1000, show_pic = True):
    # 初始化网络,在初次拟合训练时使用
    ls = []
    #迁移至GPU
    net.to(device)
    #初始 chose_Y 并选择 imf
    opimter = opim_fn(net.parameters(),lr)

    loss = torch.nn.MSELoss()
    X,Y = next(iter(data_iter))

    trans = Trans()
    y_new = trans.trans_fn(Y)
    out_time = torch.log(y_new.std() * y_new.abs().mean()/(500) + 1)

    for i in range(epoch):
        one_temp = []
        for X,Y in data_iter:
            X_new = trans.trans_fn(X)
            Y_new = trans.trans_fn(Y)
            yhat = net(X_new.to(device))
            y1 = Y_new.to(device)
            l = loss(yhat[5:],trans_shape(y1)[5:])
            if i> out_num and get_num(out_time) > get_num(l):
                return ls,trans
            opimter.zero_grad()
            l.backward()
            grad_clipping(net,1)
            opimter.step()
            one_temp.append(l)
            #ls.append(l)
        one_temp = torch.stack(one_temp,dim=0)
        ls.append(one_temp.max())

        if i % 20 == 0:
            clear_output(wait=True)
            print('out_time is loss less than',out_time,'and i is',i)
            if show_pic == True:
                plt.cla()
                temp = torch.stack(ls,dim=0)
                plt.plot(get_num(temp)[-100:])
                plt.show()

                plt.plot(get_num(trans_shape(y1))[:],'r')
                plt.plot(get_num(yhat)[:],'b',alpha=0.4)
                plt.show()
    return ls,trans

# %%
def trans_shape(Y):
    b,s,l = Y.shape
    y = Y.reshape(b*s,-1)[:,-1].reshape(-1,1)
    return y


# 使用选定的条目筛选项目
def filter_item(data, filter_list,key='供应商ID'):
    temp_data = data.copy()
    indexs = data[key]
    indexs.name=None
    temp_data.index = indexs
    output_data = temp_data.loc[filter_list,:]
    return output_data.reset_index(drop=True)

#传入一个数据框和总和,按顺序取到所有累加值,直至等于总和
def sort_and_sub(data,total):
    def cumsum_to_sub(temp_one):
        temp_index = temp_one[1:].cumsum() > temp_one['temp']
        temp_two = temp_one[1:].copy()
        temp_two.loc[temp_index] = 0
        idx_max = len(temp_two[temp_index==False])
        if idx_max < len(temp_two):
            temp_two.loc[idx_max] = temp_one['temp'] - temp_two.sum()
        return temp_two
    temp_data = data[:]
    temp_total = total[:]
    temp_total.index = temp_data.columns
    temp_total.name='temp'
    new_temp = pd.concat([pd.DataFrame(temp_total).T,temp_data],axis=0)
    temp_temp = new_temp.apply(cumsum_to_sub)
    return temp_temp

#删除Nan
def drop_nan(serise):
    temp_serise = serise.copy()
    nan_index = (pd.isna(temp_serise)!=True)
    return temp_serise.loc[nan_index]

def set_plt_size(long=12,high=8):
    plt.rcParams['figure.figsize'] = (long,high)



def predict_product(data,pred_step = 48, length = 48 , step = 24, batchSize = 12, hidden = 256 , max_epoch = 1000, min_epoch = 150, device = 'cpu',show_pic = True):
    temp_data = np.array(data,dtype=np.float32)
    dataX = DataSeq(DataSeq(temp_data[:-1],length),step)
    dataY = DataSeq(DataSeq(temp_data[1:],length),step)
    data = PureData(dataX,dataY)
    data_iter = torch.utils.data.DataLoader(data,batchSize,shuffle=True)
    X,Y = next(iter(data_iter))

    net = MyLstm_reg(length, hidden)
    ls,trans = train_begin(net,data_iter,max_epoch,device,out_num=min_epoch,show_pic=show_pic)
    new_temp = torch.tensor(temp_data[-(length+step):])
    new_temp = trans.trans_fn(new_temp).numpy()

    for i in range(pred_step):
        dataX = DataSeq(DataSeq(new_temp[:-1],length),step)
        dataY = DataSeq(DataSeq(new_temp[1:],length),step)
        data = PureData(dataX,dataY)
        data_iter = torch.utils.data.DataLoader(data,batchSize)
        X,Y = next(iter(data_iter))
        X_new = Y
        yhat = net(X_new.to(device))
        new_temp = np.concatenate((new_temp,get_num(yhat[-1])))[-(length+step):]

    one = torch.tensor(new_temp)
    two = trans.re_trans_fn(one)
    return two,ls

#常数
gongying = "附件1 近5年402家供应商的相关数据.xlsx"
data_order = pd.read_excel(gongying,'企业的订货量')
data_supply = pd.read_excel(gongying,'供应商的供货量')

device = 'cuda:0'
\end{lstlisting}

\subsection{main.py}

\begin{lstlisting}[language=python]
# %%
from header import *
import pulp
from random import choice

# %%
df_pred = pd.read_csv('pred.csv').drop(columns='Unnamed: 0')

# %%
#获取分类数据
data = pd.read_csv('temp.csv')
index = data[data['选中'] == 1]['Unnamed: 0'].values

df_class = filter_item(data_order.loc[:,['供应商ID','材料分类']],index)

new_df_pred = pd.concat([df_class['材料分类'],df_pred],axis=1)

# %%
new_df_pred.apply(re_pre_ABC,axis=1).sum(axis=0)

# %%
#new_df_pred = new_df_pred.apply(re_pre_ABC,axis=1)

# %%
new_df_pred = new_df_pred.iloc[:,:2+24]

# %%
#预测的最大供货量的和
pred_sum = new_df_pred.iloc[:,2:].sum(axis=0)
#去除供应商id和类别后预测的最大供应量
pred_pure = new_df_pred.iloc[:,2:]
#大于2.84的索引
low_index = (pred_sum -2.84e4 < 0)
#小于2.84的索引
big_index = (pred_sum -2.84e4 > 0)

# %%
Storage_cost = 0.1
gross_profit = 1
# 开始准备线性规划数据

big_num_index = pred_sum.index.to_list()

myProblem = pulp.LpProblem('订购数量规划',sense=pulp.LpMaximize)

D_w = pred_sum.copy()
D_w.loc[big_index] = 0
D_w.loc[low_index] = 2.84e4 - D_w[low_index]
D_w = D_w.tolist()

d_w_constraint = pred_sum.copy()
d_w_constraint.loc[low_index] = 0
d_w_constraint.loc[big_index] = d_w_constraint[big_index] - 2.84e4
d_w_constraint = d_w_constraint.tolist()

d_var = pulp.LpVariable.dicts(name='d_w',indices=big_num_index,lowBound=0)

u = pulp.LpVariable.dicts(name='u_w',indices=big_num_index,lowBound=0)
for i in range(len(big_num_index)):
    u[str(i)] = 2.84e4 + pulp.lpSum([d_var[str(j)] for j in range(i+1)]) - sum(D_w[:i+1])

#定义目标
myProblem += gross_profit * pulp.lpSum(d_var) - Storage_cost * pulp.lpSum(u)

#定义约束

for i in range(len(big_num_index)):
    myProblem += (d_var[str(i)] <= d_w_constraint[i])

for i in range(len(big_num_index)):
    myProblem += (u[str(i)] >= 2.84e4*2)

myProblem += pulp.lpSum(d_var) <= sum(D_w) + 2.84e4

#solve
myProblem.solve()

# %%
#检查输出
ls = []
for v in u.values():
    print(v.name, "=", v.value())
    ls.append(v.value())
print(sum(ls))

for v in d_var.values():
    print(v.name, "=", v.varValue)

# %%
ls = []
for v in d_var.values():
    #print(v.name, "=", v.varValue)
    ls.append(v.value())
#预测的供货量
pred_order = pd.Series(ls)
#预测的最大供货量的和的复制
pred_order_temp = pred_sum.copy()
#经过加上2.84e4的遮蔽运算,得到的是每周预计供货量
pred_order_temp.loc[big_index] = 2.84e4
pred_order = pd.Series(pred_order_temp.values + pred_order.values)

# %%
pred_order

# %%
#倒序排列后的预测最大供应量数据,包含供应商ID和类别
temp = new_df_pred.iloc[::-1,:].reset_index(drop=True)
temp_pure = temp.iloc[:,2:]
out = pd.concat([temp.iloc[:,:2],sort_and_sub(temp_pure,pred_order)],axis=1)
out.to_csv('订购方案.csv',index=None)



# %%
#获取分数
index = data[data['选中'] == 1]['Unnamed: 0'].values
Fraction = pd.read_csv('总表.csv').drop(columns='Unnamed: 0')
Fraction = filter_item(Fraction,index)
Fraction = Fraction['综合排分'].values

# %%
pd.read_csv('总表.csv').drop(columns='Unnamed: 0')

# %%

#获取偏差
index = data[data['选中'] == 1]['Unnamed: 0'].values

data_order = filter_item(data_order, index)
data_supply = filter_item(data_supply, index)

pure_race_data = ((data_supply.iloc[:,2:] - data_order.iloc[:,2:]) / data_order.iloc[:,2:])
pure_race_data = pure_race_data.fillna(0)

race_mean = pure_race_data.sum(axis=1) / (data_order.iloc[:,2:] !=0).sum(axis=1)
#取偏差绝对值
race_mean_abs = race_mean

pred_order_index = pred_order[(pred_order != 0)].index.to_list()
pred_order_pure = pred_pure.loc[:,[str(i) for i in pred_order_index]]
pred_order_lp = pred_order[pred_order_index]

# %%
# 开始准备线性规划数据

def get_week_order(week):

    pred_order_pure_one_week = pred_order_pure[str(week)].tolist()

    pred_order_lp_one_week = pred_order_lp[week].tolist()

    myProblem1 = pulp.LpProblem('订购数量分配规划',sense=pulp.LpMaximize)

    z_var = pulp.LpVariable.dicts(name='z_w_'+str(week)+'_n',indices=range(len(pred_order_pure_one_week)),lowBound=0)

    #定义目标
    myProblem1 += pulp.lpSum([Fraction[i] * z_var[i]  for i in range(len(pred_order_pure_one_week))])

    #定义约束

    #myProblem1 += pulp.lpSum([z_var[i]*(1/(1+race_mean[i])) for i in range(len(pred_order_pure_one_week))]) == 2.84e4 + pred_order_lp_one_week
    myProblem1 += pulp.lpSum([z_var[i]*((1+race_mean[i])) for i in range(len(pred_order_pure_one_week))]) ==pred_order_lp_one_week

    #print(pred_order_lp_one_week)

    for i in range(len(pred_order_pure_one_week)):
        myProblem1 += z_var[i]*(1/(1+race_mean[i])) <= pred_order_pure_one_week[i]

    myProblem1.solve()

    ls = []
    i=0
    for v in z_var.values():
        #print(v.name, "=", v.varValue,'pred_order = ',pred_order_pure_one_week[i],'race =',race_mean[i])
        ls.append(v.value())
        i += 1

    #print(pulp.lpSum([z_var[i]*((1+race_mean[i])) for i in range(len(pred_order_pure_one_week))]).value() )

    out = pd.Series(ls,name=f'W{week}')

    return out

# %%
out = []
for i in range(len(big_num_index)):
    if i in pred_order_index:
        out.append(get_week_order(i))
    else:
        out.append(pred_pure[str(i)].rename(f'W{i}'))

# %%
out_order = pd.DataFrame(out).T
out_order.index = data[data['选中'] == 1]['Unnamed: 0'].values

# %%
out_order.values.sum()

# %%
(out_order.values*(race_mean.values+1).reshape(44,-1)).sum()

# %%
# 添加分类数据,方便还原
out_order.insert(0,"材料分类",new_df_pred["材料分类"].values)
out_order

# %%
#temp = (out_order.values*(race_mean.values+1).reshape(44,-1))
#pd.DataFrame(temp)

# %%
# 先还原,后导出
out_order.apply(lambda x:re_pre_ABC(x,data_index=1),axis=1).to_csv('order_24_week.csv')

# %%
#获取比率new_pred_pure_race_data
index = data[data['选中'] == 1]['Unnamed: 0'].values

data_order = filter_item(data_order, index)
data_supply = filter_item(data_supply, index)

pred_pure_race_data = (data_supply.iloc[:,2:] - data_order.iloc[:,2:]) / data_order.iloc[:,2:]

new_pred_pure_race_data = pred_pure_race_data.copy()

#储存到 dict中,方便调用
supplyer, week = new_pred_pure_race_data.shape
race_dict = {}
for i in range(supplyer):
    race_dict[i] = drop_nan(new_pred_pure_race_data.loc[i] + 1).tolist()

#设置函数方便抽取随机数

def get_all_race_random():
    ls = []
    for i in race_dict.values():
        ls.append(choice(i))
    return ls

# %%
race_mean.values+1

# %%
def set_plt_size(long=12,high=8):
    plt.rcParams['figure.figsize'] = (long,high)

# %%
2**16

# %%
pd.DataFrame(out_order.values * (race_mean.values+1).reshape(44,-1)).sum(axis=0).cumsum() + 2.84e4 - pd.Series([x*2.84e4 for x in range(1,24+1)])

# %%
pd.DataFrame(out_order.values * (race_mean.values+1).reshape(44,-1)).sum(axis=0).cumsum() + 2.84e4

# %%
out_order.iloc[:,1:].values * (race_mean.values+1).reshape(44,-1)

# %%
def 动态库存量(df):
    sum_temp = df.sum(axis=0).cumsum()+2.84e4
    temp = pd.Series([x*2.84e4 for x in range(1,len(sum_temp) +1)])
    return sum_temp - temp
动态库存量(pd.DataFrame(out_order.iloc[:,1:].values * (race_mean.values+1).reshape(44,-1)))

# %%
动态库存量(pd.DataFrame(out_order.iloc[:,1:].values * (race_mean.values+1).reshape(44,-1))).values

# %%


# %%
# 绘制预测供货量与实际供货量的仿真
sum_ls  = []
set_plt_size()
for ssjds in range(32):
    weeks = out_order.columns
    simulation = []
    for week in weeks:
        race_week = get_all_race_random()
        simulation.append(out_order[week] * race_week)

    out_simulation = pd.DataFrame(simulation).T

    #out_simulation.sum(axis=0).plot(alpha=0.4,color='gray',linewidth=0.05)
    out_simulation.sum(axis=0).plot(alpha=0.1,color='gray',linewidth=0.5)

    sum_ls.append(out_simulation.sum().sum()/24)
pd.DataFrame(out_order.values * (race_mean.values+1).reshape(44,-1)).sum().plot(color='r',linewidth=2)
#plt.savefig('fig/预测仿真.png',dpi=320)

# %%
max(12,1,122)

# %%
def 动态库存量(df):
    sum_temp = df.sum(axis=0)
    length = len(sum_temp)
    temp = 2.84e4
    ls = []
    for i in range(length):
        temp = max((temp + sum_temp[i] - 2.84e4) , 0)
        ls.append(temp)
    return pd.Series(ls)
sum_ls  = []
set_plt_size()
for ssjds in range(3200):
    weeks = out_order.columns
    simulation = []
    for week in weeks:
        race_week = get_all_race_random()
        simulation.append(out_order[week] * race_week)

    out_simulation = pd.DataFrame(simulation).T

    #out_simulation.sum(axis=0).plot(alpha=0.4,color='gray',linewidth=0.05)
    #动态库存量(out_simulation).plot(alpha=0.1,color='gray',linewidth=0.5)
    动态库存量(out_simulation).plot(alpha=0.4,color='gray',linewidth=0.05)

    sum_ls.append(out_simulation.sum().sum()/24)
动态库存量(pd.DataFrame(out_order.values * (race_mean.values+1).reshape(44,-1))).plot(color='r',linewidth=2)
plt.plot([0 for i in range(24)],color='blue',linewidth=1)
plt.savefig('fig/预测库存量仿真.png',dpi=320)

# %%
动态库存量(pd.DataFrame(out_order.values * (race_mean.values+1).reshape(44,-1)))

# %%
[0 for i in range(24)]

# %%
sum_ls  = []
set_plt_size()
for ssjds in range(320):
    weeks = out_order.columns
    simulation = []
    for week in weeks:
        race_week = get_all_race_random()
        simulation.append(out_order[week] * race_week)

    out_simulation = pd.DataFrame(simulation).T

    #out_simulation.sum(axis=0).plot(alpha=0.4,color='gray',linewidth=0.05)
    out_simulation.sum(axis=0).plot(alpha=0.1,color='gray',linewidth=0.5)

    sum_ls.append(out_simulation.sum().sum()/24)
pd.DataFrame(out_order.values * (race_mean.values+1).reshape(44,-1)).sum().plot(color='r',linewidth=2)
plt.savefig('fig/预测仿真.png',dpi=320)

# %%
(pd.DataFrame(out_order.values * (race_mean.values+1).reshape(44,-1)).sum().sum())/24

# %%
(sum(sum_ls))/320



\end{lstlisting}

\subsection{work1.py}

\begin{lstlisting}[language=python]
# %%
import numpy as np
import pandas as pd
import matplotlib.pyplot as plt

# %%
def get_bool(data,bool):
    return data[bool]
def sort_and_plot(data):
    return data.sort_values().reset_index(drop=True).plot()

# %%
#归一化方案
def Normal(x):
    return (x-x.mean()) / x.std()
def Min_Max(x):
    return (x-x.min()) / (x.max() - x.min())

# %%
def plot_and_save(data:pd.DataFrame, path:str, size=(24.0, 16.0)):
    plt.rcParams['figure.figsize'] = size
    sort_and_plot(data)
    plt.savefig(path)
    plt.cla()

# %%
gongying = "附件1 近5年402家供应商的相关数据.xlsx"
转运商 = "附件2 近5年8家转运商的相关数据.xlsx"

# %%
data_order = pd.read_excel(gongying,'企业的订货量')
data_supply = pd.read_excel(gongying,'供应商的供货量')

# %%
test_data = data_order.copy()

# %%
data_item_num = test_data.iloc[:,2:]

# %%
test_data['订货次数']=(data_item_num>0).sum(axis=1)
test_data['订货总量'] = data_item_num.sum(axis=1)
test_data['供货总量'] = data_supply.iloc[:,2:].sum(axis=1)

# %%
data_sub = data_item_num-data_supply.iloc[:,2:]

# %%
test_data['平均供货偏差'] = (((data_sub / data_item_num).abs().fillna(0)).sum(axis=1) / test_data['订货次数'])
test_data['单次最大供应量'] = data_supply.iloc[:,2:].max(axis=1)

# %%
test_data.iloc[:,-5:]

# %%
targets = test_data.columns[-5:]

# %%
for target in targets:
    plot_and_save(test_data[target],'fig/'+target+'.png',size=(12.0,8))

# %%
data_change = test_data.iloc[:,-5:]

# %%
for i in [ '订货总量', '供货总量', '单次最大供应量']:
    data_change[i] = np.log(data_change[i].values)

# %%
data_change = data_change.apply(Min_Max)

# %%
data_change

# %%
data_temp = np.asarray(data_change[['订货次数', '订货总量', '供货总量', '平均供货偏差', '单次最大供应量']])
#计算熵值
k = -1/np.log(402)
data_log= data_temp*np.log(data_temp)
data_log = pd.DataFrame(data_log)
data_log=data_log.fillna(0)
data_log=data_log.values
ls=[]
#计算变异指数
for i in range(5):
    e_j=k*data_log.sum(axis=0)[i]
    ls.append(e_j)
temp_list =[]
for i in ls:
    temp_list.append(1-i)
#计算权重
ls=[]
#删除错误定义
#del(sum)
for i in temp_list:
    ls.append(i/sum(temp_list))


# %%
print(ls,targets)

\end{lstlisting}

\subsection{linear.jl}

\begin{lstlisting}
### A Pluto.jl notebook ###
# v0.19.9

using Markdown
using InteractiveUtils

# This Pluto notebook uses @bind for interactivity. When running this notebook outside of Pluto, the following 'mock version' of @bind gives bound variables a default value (instead of an error).
macro bind(def, element)
    quote
        local iv = try Base.loaded_modules[Base.PkgId(Base.UUID("6e696c72-6542-2067-7265-42206c756150"), "AbstractPlutoDingetjes")].Bonds.initial_value catch; b -> missing; end
        local el = $(esc(element))
        global $(esc(def)) = Core.applicable(Base.get, el) ? Base.get(el) : iv(el)
        el
    end
end

# ╔═╡ 5928cfb0-1923-11ed-3ace-2bfad8956c7c
using JuMP,Gurobi,DataFrames,CSV,Parsers,Tables,PlutoUI

# ╔═╡ 274a1543-f320-4705-8e98-55d6a3fbea74
function Base.filter((title,name)::Tuple, df::DataFrame)
	return filter(title=>x -> x == name, df)
end

# ╔═╡ dda52e10-674b-47ab-b765-aa4acf90780a
function Base.filter((title,name)::Tuple, df::DataFrame,week::Integer)
	return filter(title=>x -> x == name, df)[:,"W"*string(week-1)]
end

# ╔═╡ eb6d766a-bc18-4f21-9945-d2bb387a5e6d
function Base.filter((title,name)::Tuple, df::DataFrame,week::String)
	return filter(title=>x -> x == name, df)[:,week]
end

# ╔═╡ d2e2fc85-3b17-4fce-af9b-5f1d38109691
function Base.sum(l::Vector{Vector{AffExpr}})
	temp = zero(l[1][1])
	for i in l
		temp += sum(i)
	end
	return temp
end

# ╔═╡ 243d047b-920d-451c-946e-1fcd45f227b4
function num2weekStr(num)
	return "W"*string(num-1)
end

# ╔═╡ 2fd96ad3-0367-49ea-846e-07f6722f82ff
@bind week Slider(1:24)

# ╔═╡ 36ca59d9-2891-43af-bdbb-0a2a4eee5853
md"""
 目前在计算的周: $(week)
"""

# ╔═╡ 9cea8136-3a62-4776-ba94-0f1aa0eaaf42
weekStr = num2weekStr(week)

# ╔═╡ 500d4a9c-0879-4738-bf22-ad726d90fe9d
@bind limit Slider(1:5)

# ╔═╡ 7ec11938-e35a-478c-9af3-5b8a8e0108b6
md"""
目前的转运商允许量:$(limit)
"""

# ╔═╡ 4669b138-a94b-43f2-a4c2-0a803f1ca3d0
@bind Transshipment_capacity Slider(6000:500:22000)

# ╔═╡ b1530cee-90e3-483d-86e8-cc54e7c24338
md"""
目前的转运能力:$(Transshipment_capacity)
"""

# ╔═╡ a9ed9185-630c-4c6e-aac5-155900de98f3
md"""
## 导入数据
"""

# ╔═╡ 4c41b978-fdd4-4b03-b621-f0753b727eb5
begin
	df_order_24_week = CSV.read("order_24_week (2).csv",DataFrame)
	df_Average_loss_of_forwarders = CSV.read("转运商平均损耗.csv",DataFrame)
	forwarders_id = df_Average_loss_of_forwarders[:,:转运商ID]
end

# ╔═╡ ef66f393-4a3b-4d0a-8d20-dd521b5529cd
filter((:材料分类,"C"),df_order_24_week,weekStr)

# ╔═╡ ccb7017d-eaaf-451e-96f0-76b729ba799c
md"""
# 线性方程求解函数
"""

# ╔═╡ 7f084915-1fb5-46aa-bc15-e1d73ed5960e
function solve_chose_forwarders(week::String,
								df_order::DataFrame,
								df_Average_loss::DataFrame,
								rate_list::Vector,
								limit::Int64,
								Transshipment_capacity::Int64)
	#每个材料类别的损耗量
	function sum_model(uⁱ,
						len_of_u::Int64,
						num_of_forwarder::Int64,
						Average_loss::Vector{Float64},
						order::Vector{Float64})
		sum_i = sum([[uⁱ[i,j] * Average_loss[j] * order[i] for i in 1:len_of_u] for j in 1:num_of_forwarder])
		return sum_i
	end

	# 全部损耗量
	function sum_model(U::Vector,
						length_list::Vector,
						num_of_forwarder::Int64,
						Average_loss::Vector{Float64},
						order_list::Vector{Vector{Float64}},
						rate_list::Vector)
		temp=0
		for (uⁱ,len_of_u,order,rate) in zip(U,length_list,order_list,rate_list)
			temp += rate* sum_model(uⁱ,
						len_of_u,
						num_of_forwarder,
						Average_loss,
						order)
		end
		return temp
	end

	#每一个供应商的转运量
	function sum_forwarders(index_of_forwarders::Int64,
							U::Vector,
							length_list::Vector,
							Average_loss::Vector{Float64},
							order_list::Vector{Vector{Float64}})
		temp = 0
		j = index_of_forwarders
		for (uⁱ,len_of_u,order) in zip(U,length_list,order_list)
			temp += sum([uⁱ[i,j] * order[i] for i in 1:len_of_u])
		end
		return temp
	end

	# 01约束
	function sum_bin(uⁱ,len::Int64,model)
		for i in 1:len
			@constraint(model, sum(uⁱ[i,:])<=limit)
		end
	end

	#运货量等于订货量
	function sum_bin(uⁱ,len::Int64,
					model,
					order::Vector{Float64},
					num_of_forwarder::Int64)
		for i in 1:len
			@constraint(model, sum([uⁱ[i,j] * order[i] for j in 1:num_of_forwarder]) == order[i])
		end
	end

	#定义模型
	model = Model(Gurobi.Optimizer)

	#定义变量
	num_of_forwarder=length(df_Average_loss_of_forwarders[:,1])
	begin
		length_A = length(filter((:材料分类,"A"),df_order,week))
		length_B = length(filter((:材料分类,"B"),df_order,week))
		length_C = length(filter((:材料分类,"C"),df_order,week))
		@variable(model, uᴬ[i = 1:length_A, j= 1:num_of_forwarder],Bin)
		@variable(model, uᴮ[i = 1:length_B, j= 1:num_of_forwarder],Bin)
		@variable(model, uᶜ[i = 1:length_C, j= 1:num_of_forwarder],Bin)
	end

	#定义目标函数
	@objective(model, Min,
			sum_model([uᴬ,uᴮ,uᶜ],
				[length_A,length_B,length_C],
				num_of_forwarder,
				df_Average_loss[:,:平均损耗],
				[filter((:材料分类,"A"),df_order,week),
					filter((:材料分类,"B"),df_order,week),
					filter((:材料分类,"C"),df_order,week)],
				rate_list))

	#定义约束
	#01约束
	for (uⁱ,len_of_u) in zip([uᴬ,uᴮ,uᶜ],[length_A,length_B,length_C])
		sum_bin(uⁱ,len_of_u,model)
	end
	#供应商的转运量小于6000
	for j in 1:num_of_forwarder
		@constraint(model,
		sum_forwarders(j,[uᴬ,uᴮ,uᶜ],[15,14,15],
					df_Average_loss[:,:平均损耗],
					[filter((:材料分类,"A"),df_order,week),
					filter((:材料分类,"B"),df_order,week),
					filter((:材料分类,"C"),df_order,week)])
		<=Transshipment_capacity)
	end
	#运货量等于订货量
	for (uⁱ,len_of_u,order) in zip([uᴬ,uᴮ,uᶜ],
							[length_A,length_B,length_C],
							[filter((:材料分类,"A"),df_order,week),
								filter((:材料分类,"B"),df_order,week),
								filter((:材料分类,"C"),df_order,week)])
		sum_bin(uⁱ,len_of_u,model,order,num_of_forwarder)
	end
	optimize!(model)
	return model,(uᴬ,uᴮ,uᶜ)
end

# ╔═╡ 407fe9e3-c524-4b51-a0dd-0229e398f535
function solve_chose_forwarders(week::String,
								df_order::DataFrame,
								df_Average_loss::DataFrame)
	solve_chose_forwarders(weekStr,df_order,df_Average_loss,[1.2,1.1,1,1,1],limit,Transshipment_capacity)
end

# ╔═╡ 3737fa99-4b4d-46f3-bdd4-106e589aa847
(model,var) = solve_chose_forwarders(weekStr,df_order_24_week,df_Average_loss_of_forwarders)

# ╔═╡ febc35df-f986-43e6-a98a-6a8107ef7dd9
termination_status(model)

# ╔═╡ 1f51645c-fa66-45d0-a2a3-846554264525
termination_status(model)

# ╔═╡ 14a0e20c-b79d-4d0f-aed9-e31753f2ddfa
md"""
# 把解转化为坐标
"""

# ╔═╡ 02bd0f1c-4c41-4a3a-8928-d578cccec15a
# 解矩阵转化为数字
function mat2sym(mat::Matrix)

	function var2sym(sym_list::Vector,
					 var::Vector)
		temp = missing
		for (i,sym) in zip(var,sym_list)
			if i == 1.0
				temp = sym
				break
			end
		end
		return temp
	end

	function mat2vec(mat::Matrix)
		x,_ = size(mat)
		return [mat[i,:] for i in 1:x]
	end

	return mat |> x->value.(x) |> mat2vec |> x->(x-> var2sym(forwarders_id,x)).(x)
end

# ╔═╡ 9851ee35-f131-48e7-b87e-1c03d27b5263
# 解转化为坐标数字元组
function solve2location(供应商ID表_list::Vector,
						solve_list::Tuple,
						data_list::Vector)
	#辅助函数
	function delete(str::AbstractString,
					del::String)
		return replace(str,del=>"")
	end

	function sym2num(str::AbstractString)
		return str |> x->delete(x,"T") |> x->delete(x,"S") |> x->Parsers.parse(Int64,x)
	end

	function sym2num(str::Missing)
		return -1
	end

	function mat2location(供应商ID表::Vector,
							solve::Matrix,
							data::Vector)
		x = 供应商ID表 |> x->sym2num.(x)
		y = solve |> mat2sym |> x->sym2num.(x)
		return zip(x,y,data)
	end

	temp = []
	for (供应商ID表,solve,data) in zip(供应商ID表_list,solve_list,data_list)
		output = mat2location(供应商ID表,solve,data)
		for i in output
			temp = cat(temp,i,dims=1)
		end
	end
	return temp
end

# ╔═╡ a083a5da-0536-4bd9-9c5f-2ec66678d7f2
begin
	供应商ID表_list = [filter((:材料分类,i),df_order_24_week)[:,:Column1] for i in ["A","B","C"]]
	周订购数据_list = [filter((:材料分类,i),df_order_24_week)[:,week+2] for i in ["A","B","C"]]
	location_list = solve2location(供应商ID表_list, var, 周订购数据_list)
end

# ╔═╡ 393930a9-26f6-4a25-9648-97b90d21eb51
md"""
# 写入表格,准备复制
"""

# ╔═╡ c9dee99e-9a55-4092-b2e4-e48dc1dcccba
#temp = DataFrame(fill!(Matrix{Float64}(undef, 402, 8),-114.514), :auto)

# ╔═╡ c7764ec3-6fc2-455a-8fdf-b1a319d4c0f6
for i in location_list
	(y,x,data) = i
	if x == -1
		continue
	end
	temp[y,x] =data
end

# ╔═╡ a9df1cd9-bfe1-461d-bac3-a91589c2163f
CSV.write("temp_location-$(weekStr).csv",temp)

# ╔═╡ 7e218375-87e3-44a7-afa3-d5f4052375fc
#read("temp_location-$(weekStr).csv",String) |> x->replace(x,"-114.514"=>"") |> x-> write("temp_location-$(weekStr).csv",x)

# ╔═╡ 606088ae-d999-42ea-aa98-308412ac779d
function write_table(location_list::Vector,weekStr::String)
	temp = DataFrame(fill!(Matrix{Float64}(undef, 402, 8),-114.514), :auto)
	for i in location_list
		(y,x,data) = i
		if x == -1
			continue
		end
		temp[y,x] =data
	end
	path = joinpath("location","temp_location-$(weekStr).csv")
	CSV.write(path,temp)
	read(path,String) |> x->replace(x,"-114.514"=>"") |> x-> write(path,x)
end

# ╔═╡ 2927eb71-9b9e-4026-8857-3331e715fc39
joinpath("location","temp_location-$(weekStr).csv")

# ╔═╡ 618e342c-f035-4337-8705-91c10eaffbfd
write_table(location_list,weekStr)


\end{lstlisting}
\end{appendices}