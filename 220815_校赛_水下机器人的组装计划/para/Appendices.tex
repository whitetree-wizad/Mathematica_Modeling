



\section{模型的评价与推广} % (fold)
\label{sec:模型的评价与推广}


\subsection{灵敏度分析} % (fold)
\label{sub:灵敏度分析}

灵敏度分析是指对系统或事物因周围条件变化显示出来的敏感程度的分析。 在前文讨论的线性规划模型中, 都假定问题中的数据为常数。
但实际上, 这些参数往往是一些估计或预测的数字, 经常有少许的变动。 因此就会产生以下问题:
\begin{enumerate}
    \item 当这些参数中的一个或几个发生变化时,问题的最优解会有什么变化?
    \item 这些参数的变化限制在什么范围内,问题的原最优解不变。
\end{enumerate}


但本文在实际应用中,是使用计算机求解的,有数值方法:给定参变量一个步长,使其重复求解线性规划问题,以观察最优解的变化情况。
结果表明模型对库存费的灵敏度最高、生产准备费次之(其中WPCR的生产准备费又在其中最大)。
但模型对其他参变量并不敏感。总体来看,本文认为本模型可帮助工厂根据情况选择合适的生产计划制定策略。
% subsection 灵敏度分析 (end)

\subsection{模型的创新点} % (fold)
\label{sub:模型的创新点}

\begin{enumerate}
    \item 本文通过对约束条件进行变换,维持了优化模型的比例性和确定性,从而使计算机方便处理,保留作为线性规划具有确定全局最优解的良好性质。
    \item 引入了0$\textendash$1变量解决停机检修问题,容易理解又便于计算。
\end{enumerate}

% subsection 模型的创新点 (end)

\subsection{值得改进的部分} % (fold)
\label{sub:值得改进的部分}


\begin{enumerate}
    \item 问题3中计算过于复杂,仅得到了局部最优解,应进一步简化算法的时间复杂度,提高优化效率。
    \item 问题4仅以预测概率代替正常交付概率,使预测和优化分离。欲进一步考虑博弈概率模型进行多目标规划。
\end{enumerate}

% subsection 值得改进的部分 (end)

% section 模型的评价与推广 (end)


\printbibliography
\newpage
%附录
\begin{appendices}

\section{源程序}


\subsection{ELOL.py} % (fold)
% subsection elol (end)
\begin{lstlisting}[language=python]

\end{lstlisting}

\subsection{work1.py}

\begin{lstlisting}[language=python]


\end{lstlisting}

\subsection{linear.jl}

\begin{lstlisting}

\end{lstlisting}
\end{appendices}