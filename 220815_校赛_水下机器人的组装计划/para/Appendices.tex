



\section{模型的评价与推广} % (fold)
\label{sec:模型的评价与推广}


\subsection{灵敏度分析} % (fold)
\label{sub:灵敏度分析}

灵敏度分析是指对系统或事物因周围条件变化显示出来的敏感程度的分析。 在前文讨论的线性规划模型中, 都假定问题中的数据为常数。
但实际上, 这些参数往往是一些估计或预测的数字, 经常有少许的变动。 因此就会产生以下问题:
\begin{enumerate}
    \item 当这些参数中的一个或几个发生变化时,问题的最优解会有什么变化?
    \item 这些参数的变化限制在什么范围内,问题的原最优解不变。
\end{enumerate}


但本文在实际应用中,是使用计算机求解的,有数值方法:给定参变量一个步长,使其重复求解线性规划问题,以观察最优解的变化情况。
结果表明模型对库存费的灵敏度最高、生产准备费次之(其中WPCR的生产准备费又在其中最大)。
但模型对其他参变量并不敏感。总体来看,本文认为本模型可帮助工厂根据情况选择合适的生产计划制定策略。
% subsection 灵敏度分析 (end)

\subsection{模型的创新点} % (fold)
\label{sub:模型的创新点}

\begin{enumerate}
    \item 本文通过对约束条件进行变换,维持了优化模型的比例性和确定性,从而使计算机方便处理,保留作为线性规划具有确定全局最优解的良好性质。
    \item 引入了0$\textendash$1变量解决停机检修问题,容易理解又便于计算。
\end{enumerate}

% subsection 模型的创新点 (end)

\subsection{值得改进的部分} % (fold)
\label{sub:值得改进的部分}


\begin{enumerate}
    \item 问题3中计算过于复杂,仅得到了局部最优解,应进一步简化算法的时间复杂度,提高优化效率。
    \item 问题4仅以预测概率代替正常交付概率,使预测和优化分离。欲进一步考虑博弈概率模型进行多目标规划。
\end{enumerate}

% subsection 值得改进的部分 (end)

% section 模型的评价与推广 (end)


\printbibliography
\newpage
%附录
\begin{appendices}

\section{源程序}


\subsection{水下机器人的组装计划.jl} % (fold)
% subsection elol (end)
\begin{lstlisting}[language=julia]
    using Markdown
    using InteractiveUtils
    
    using JSON, JuMP, DataFrames, CSV, Gurobi
    
    function read_data(path::AbstractString)
        return CSV.read(path, DataFrame; transpose=true)
    end
    
    begin
        生产准备费用和单件库存费用() = read_data("每次生产准备费用和单件库存费用.csv")
        WPCR需求和关键设备工时限制() = read_data("每天 WPCR 需求和关键设备工时限制.csv")
        连续30周的WPCR需求数据() = read_data("连续 30 周的 WPCR 需求数据.csv")
    end
    
    WPCR需求和关键设备工时限制()
    
    连续30周的WPCR需求数据()
    
    struct Component
        生产准备费用::Float64
        单件库存费用::Float64
        begin
            function Component(生产准备费用::Float64, 单件库存费用::Float64)
                if 生产准备费用 < 0
                    throw(error("生产准备费用不能为负数!"))
                end
                if 单件库存费用 < 0
                    throw(error("单件库存费用不能为负数!"))
                end
                return new(生产准备费用, 单件库存费用)
            end
    
            function Component(生产准备费用::Number, 单件库存费用::Number)
                return Component(Float64(生产准备费用), Float64(单件库存费用))
            end
        end
    end
    
    struct Element
        生产准备费用::Float64
        单件库存费用::Float64
        工时消耗::Int64
        组件::Dict{String,Component}
        需求量::Dict{String,Int64}
        function Element(生产准备费用::Number, 单件库存费用::Number, 工时消耗::Int64, 组件::Dict{String,Component}, 需求量::Dict{String,Int64})
            if 生产准备费用 < 0
                throw(error("生产准备费用不能为负数!"))
            end
            if 单件库存费用 < 0
                throw(error("单件库存费用不能为负数!"))
            end
            return new(Float64(生产准备费用), Float64(单件库存费用), 工时消耗, 组件, 需求量)
        end
    end
    
    struct Product
        生产准备费用::Float64
        单件库存费用::Float64
        组件::Dict{String,Element}
        需求量::Dict{String,Int64}
        function Product(生产准备费用::Number, 单件库存费用::Number, 组件::Dict{String,Element}, 需求量::Dict{String,Int64})
            if 生产准备费用 < 0
                throw(error("生产准备费用不能为负数!"))
            end
            if 单件库存费用 < 0
                throw(error("单件库存费用不能为负数!"))
            end
            return new(Float64(生产准备费用), Float64(单件库存费用), 组件, 需求量)
        end
    end
    
    struct Robot_data
        生产总工时限制::Dict{String,Int64}
        需求::Dict{String,Int64}
        产品::Product
    end
    
    function product_Element(
        df::DataFrame,
        element_name::String,
        需求量::Dict{String,Int64},
        工时消耗::Int64)
    
        function get_index(df::DataFrame, name)
            filter(:产品 => x -> x == name, df)[1, :]
        end
    
        component = Dict(let item = get_index(df, key)
            String(item[:产品]) => Component(item[:生产准备费用],
                item[:单件库存费用])
        end
                         for key in keys(需求量))
        element = let item = get_index(df, element_name)
            Element(item[:生产准备费用], item[:单件库存费用], 工时消耗, component, 需求量)
        end
    
        return element
    end
    
    function product_Product(
        df::DataFrame,
        product_name::String,
        需求量::Dict{String,Int64},
        组件需求量::Dict{String,Dict{String,Int64}},
        工时消耗::Dict{String,Int64})
    
        function get_index(df::DataFrame, name)
            filter(:产品 => x -> x == name, df)[1, :]
        end
        element = Dict(key => product_Element(df, key, 组件需求量[key], 工时消耗[key])
                       for key in keys(需求量))
    
        product = let item = get_index(df, product_name)
            Product(item[:生产准备费用], item[:单件库存费用], element, 需求量)
        end
        return product
    end
    
    WPCR需求和关键设备工时限制()
    
    function solve_robot(
        Optimizer,
        data,
        add_robot_variables,
        add_robot_constraints,
        add_robot_objective,
        TimeLimit=1)
    
        model = Model(Optimizer)
        set_optimizer_attribute(model, "TimeLimit", TimeLimit)
        add_robot_variables(model, data)
        add_robot_constraints(model, data)
        add_robot_objective(model, data)
    
        optimize!(model)
        println(termination_status(model))
        if termination_status(model) != OPTIMAL
            @warn("模型未能求解!")
            return model
        end
    
        return model
    end
    
    function add_robot_variables(model, data)
    
        time = length(data.:需求)
    
        get_element(data) = data.:产品.:组件
        get_component(element) = element.:组件
    
        product = data.:产品
        element = get_element(data)
        components = [get_component(v) for (k, v) in element]
    
        let element_key = keys(element),
            component_key = keys.(components) |> x -> [[i for i in v] for v in x] |> x -> vcat(x...)
    
            @variable(model, x[["WPCR", element_key..., component_key...], [i for i in 1:time]] >= 0, Int)
            @variable(model, y[["WPCR", element_key..., component_key...], [i for i in 1:time]] >= 0, Int)
            @variable(model, ω[["WPCR", element_key..., component_key...], [i for i in 1:time]], Bin)
            @variable(model, M[[element_key...], [i for i in 1:time]] >= 0, Int)
        end
        return model, data
    end
    
    function add_robot_constraints(model, data)
        time = length(data.:需求)
    
        int2week(int) = ["周一", "周二", "周三", "周四", "周五", "周六", "周日"][int]
        get_element(data) = data.:产品.:组件
        get_component(element) = element.:组件
    
        product = data.:产品
        element = get_element(data)
        components = [get_component(v) for (k, v) in element]
    
        let element_key = keys(element), r⁻ = "WPCR"
            for t in 1:time
                for r in element_key
                    let item = element[r], r⁻ = r, component_key = keys(item.:组件)
                        for r in component_key
                            if t == 1
                                @constraint(model, model.obj_dict[:y][r, t] == 0 + model.obj_dict[:x][r, t] - item.:需求量[r] * model.obj_dict[:x][r⁻, t])
                            elseif t == 7
                                @constraint(model, model.obj_dict[:y][r, t] == model.obj_dict[:y][r, t-1] + model.obj_dict[:x][r, t] - item.:需求量[r] * model.obj_dict[:x][r⁻, t])
    
                                @constraint(model, model.obj_dict[:y][r, t] == 0)
                            else
                                @constraint(model, model.obj_dict[:y][r, t] == model.obj_dict[:y][r, t-1] + model.obj_dict[:x][r, t] - item.:需求量[r] * model.obj_dict[:x][r⁻, t])
                            end
    
                            @constraint(model, model.obj_dict[:x][r, t] == model.obj_dict[:x][r, t] * model.obj_dict[:ω][r, t])
                        end
    
                    end
    
                    if t == 1
                        @constraint(model, model.obj_dict[:y][r, t] == 0 + model.obj_dict[:x][r, t] - product.:需求量[r] * model.obj_dict[:x][r⁻, t])
                    elseif t == 7
                        @constraint(model, model.obj_dict[:y][r, t] == model.obj_dict[:y][r, t-1] + model.obj_dict[:x][r, t] - product.:需求量[r] * model.obj_dict[:x][r⁻, t])
    
                        @constraint(model, model.obj_dict[:y][r, t] == 0)
                    else
                        @constraint(model, model.obj_dict[:y][r, t] == model.obj_dict[:y][r, t-1] + model.obj_dict[:x][r, t] - product.:需求量[r] * model.obj_dict[:x][r⁻, t])
                    end
    
                    @constraint(model, model.obj_dict[:x][r, t] * element[r].:工时消耗 == model.obj_dict[:M][r, t] * model.obj_dict[:ω][r, t])
    
                    @constraint(model, model.obj_dict[:x][r, t] == model.obj_dict[:x][r, t] * model.obj_dict[:ω][r, t])
                end
    
                let r = "WPCR"
                    if t == 1
                        @constraint(model, data.:需求[int2week(t)] == 0 + model.obj_dict[:x][r, t] - model.obj_dict[:y][r, t])
                    elseif t == 7
                        @constraint(model, data.:需求[int2week(t)] == model.obj_dict[:y][r, t-1] + model.obj_dict[:x][r, t] - model.obj_dict[:y][r, t])
    
                        @constraint(model, model.obj_dict[:y][r, t] == 0)
                    else
                        @constraint(model, data.:需求[int2week(t)] == model.obj_dict[:y][r, t-1] + model.obj_dict[:x][r, t] - model.obj_dict[:y][r, t])
                    end
    
                    @constraint(model, model.obj_dict[:x][r, t] == model.obj_dict[:x][r, t] * model.obj_dict[:ω][r, t])
                end
    
                @constraint(model, sum(model.obj_dict[:M][r, t] for r in element_key) <= data.:生产总工时限制[int2week(t)])
            end
        end
        return model, data
    end
    
    function add_robot_objective(model, data)
        time = length(data.:需求)
    
        int2week(int) = ["周一", "周二", "周三", "周四", "周五", "周六", "周日"][int]
        get_element(data) = data.:产品.:组件
        get_component(element) = element.:组件
    
        product = data.:产品
        element = get_element(data)
        components = [get_component(v) for (k, v) in element]
    
        excp = 0
        let element_key = keys(element), r⁻ = "WPCR"
            for t in 1:time
                for r in element_key
                    let item = element[r], r⁻ = r, component_key = keys(item.:组件)
                        for r in component_key
                            excp += item.:组件[r].生产准备费用 * model.obj_dict[:ω][r, t] + item.:组件[r].单件库存费用 * model.obj_dict[:y][r, t]
                        end
                    end
                    excp += product.:组件[r].生产准备费用 * model.obj_dict[:ω][r, t] + product.:组件[r].单件库存费用 * model.obj_dict[:y][r, t]
                end
                let r = "WPCR"
                    excp += product.生产准备费用 * model.obj_dict[:ω][r, t] + product.单件库存费用 * model.obj_dict[:y][r, t]
                end
            end
        end
        @objective(model, Min, excp)
        return model, data
    end
    
    function get_生产准备费用(model, data, time)
    
        int2week(int) = ["周一", "周二", "周三", "周四", "周五", "周六", "周日"][int]
        get_element(data) = data.:产品.:组件
        get_component(element) = element.:组件
    
        product = data.:产品
        element = get_element(data)
        components = [get_component(v) for (k, v) in element]
    
        excp = 0
        t = time
        let element_key = keys(element), r⁻ = "WPCR"
            for r in element_key
                let item = element[r], r⁻ = r, component_key = keys(item.:组件)
                    for r in component_key
                        excp += item.:组件[r].生产准备费用 * model.obj_dict[:ω][r, t]
                    end
                end
                excp += product.:组件[r].生产准备费用 * model.obj_dict[:ω][r, t]
            end
            let r = "WPCR"
                excp += product.生产准备费用 * model.obj_dict[:ω][r, t]
            end
        end
        return excp
    end
    
    function get_生产库存费用(model, data, time)
    
        int2week(int) = ["周一", "周二", "周三", "周四", "周五", "周六", "周日"][int]
        get_element(data) = data.:产品.:组件
        get_component(element) = element.:组件
    
        product = data.:产品
        element = get_element(data)
        components = [get_component(v) for (k, v) in element]
    
        excp = 0
        t = time
        let element_key = keys(element), r⁻ = "WPCR"
            for r in element_key
                let item = element[r], r⁻ = r, component_key = keys(item.:组件)
                    for r in component_key
                        excp += item.:组件[r].单件库存费用 * model.obj_dict[:y][r, t]
                    end
                end
                excp += product.:组件[r].单件库存费用 * model.obj_dict[:y][r, t]
            end
            let r = "WPCR"
                excp += product.单件库存费用 * model.obj_dict[:y][r, t]
            end
        end
        return excp
    end
    
    md"""
    ## 第一题
    """
    
    function answer2df(model, data)
        answser = value.(model.obj_dict[:x])
        answser = answser |> Matrix |> transpose |> x -> DataFrame(x, axes(answser)[1])
        answser[!, "生产准备费用"] = (x -> get_生产准备费用(model, data, x)).([1, 2, 3, 4, 5, 6, 7]) .|> value
        answser[!, "生产库存费用"] = (x -> get_生产库存费用(model, data, x)).([1, 2, 3, 4, 5, 6, 7]) .|> value
        return answser
    end
    
    let data = let data = WPCR需求和关键设备工时限制(), (len, _) = size(data)
            let 生产总工时限制 = Dict((
                    let item = data[i, :]
                        String(item[:天]) => item["A、B、C生产总工时限制(工时)"]
                    end
                    for i in 1:len
                )),
                WPCR需求 = Dict(let item = data[i, :]
                    String(item[:天]) => item["WPCR需求(个)"]
                end
                              for i in 1:len),
                需求量 = Dict(
                    "A" => 3,
                    "B" => 4,
                    "C" => 5),
                组件需求量 = Dict(
                    "A" => Dict(
                        "A1" => 6,
                        "A2" => 8,
                        "A3" => 2),
                    "B" => Dict(
                        "B1" => 2,
                        "B2" => 4),
                    "C" => Dict(
                        "C1" => 8,
                        "C2" => 2,
                        "C3" => 12)),
                工时消耗 = Dict(
                    "A" => 3,
                    "B" => 5,
                    "C" => 5),
                product = product_Product(生产准备费用和单件库存费用(),
                    "WPCR",
                    需求量,
                    组件需求量,
                    工时消耗)
    
                Robot_data(
                    生产总工时限制,
                    WPCR需求,
                    product)
            end
        end,
        model = solve_robot(
            Gurobi.Optimizer,
            data,
            add_robot_variables,
            add_robot_constraints,
            add_robot_objective)
    
        CSV.write("第一题.csv", answer2df(model, data))
    end
    
    md"""
    ## 第二题
    """
    
    function add_robot_constraints_2(model, data)
        time = length(data.:需求)
    
        int2week(int) = ["周一", "周二", "周三", "周四", "周五", "周六", "周日"][int]
        get_element(data) = data.:产品.:组件
        get_component(element) = element.:组件
    
        product = data.:产品
        element = get_element(data)
        components = [get_component(v) for (k, v) in element]
    
        let element_key = keys(element), r⁻ = "WPCR"
            for t in 1:time
                for r in element_key
                    let item = element[r], r⁻ = r, component_key = keys(item.:组件)
                        for r in component_key
                            if t == 1
                                @constraint(model, model.obj_dict[:y][r, t] == 0 + model.obj_dict[:x][r, t] - item.:需求量[r] * model.obj_dict[:x][r⁻, t])
    
                                @constraint(model, item.:需求量[r] * model.obj_dict[:x][r⁻, t] <= 0)
                            elseif t == 7
                                @constraint(model, model.obj_dict[:y][r, t] == model.obj_dict[:y][r, t-1] + model.obj_dict[:x][r, t] - item.:需求量[r] * model.obj_dict[:x][r⁻, t])
    
                                @constraint(model, model.obj_dict[:y][r, t] == 0)
    
                                @constraint(model, item.:需求量[r] * model.obj_dict[:x][r⁻, t] <= model.obj_dict[:y][r, t-1])
                            else
                                @constraint(model, model.obj_dict[:y][r, t] == model.obj_dict[:y][r, t-1] + model.obj_dict[:x][r, t] - item.:需求量[r] * model.obj_dict[:x][r⁻, t])
    
                                @constraint(model, item.:需求量[r] * model.obj_dict[:x][r⁻, t] <= model.obj_dict[:y][r, t-1])
                            end
    
                            @constraint(model, model.obj_dict[:x][r, t] == model.obj_dict[:x][r, t] * model.obj_dict[:ω][r, t])
                        end
    
                    end
    
                    if t == 1
                        @constraint(model, model.obj_dict[:y][r, t] == 0 + model.obj_dict[:x][r, t] - product.:需求量[r] * model.obj_dict[:x][r⁻, t])
    
                        @constraint(model, product.:需求量[r] * model.obj_dict[:x][r⁻, t] <= 0)
                    elseif t == 7
                        @constraint(model, model.obj_dict[:y][r, t] == model.obj_dict[:y][r, t-1] + model.obj_dict[:x][r, t] - product.:需求量[r] * model.obj_dict[:x][r⁻, t])
    
                        @constraint(model, model.obj_dict[:y][r, t] == 0)
    
                        @constraint(model, product.:需求量[r] * model.obj_dict[:x][r⁻, t] <= model.obj_dict[:y][r, t-1])
                    else
                        @constraint(model, model.obj_dict[:y][r, t] == model.obj_dict[:y][r, t-1] + model.obj_dict[:x][r, t] - product.:需求量[r] * model.obj_dict[:x][r⁻, t])
    
                        @constraint(model, product.:需求量[r] * model.obj_dict[:x][r⁻, t] <= model.obj_dict[:y][r, t-1])
                    end
    
                    @constraint(model, model.obj_dict[:x][r, t] * element[r].:工时消耗 == model.obj_dict[:M][r, t] * model.obj_dict[:ω][r, t])
    
                    @constraint(model, model.obj_dict[:x][r, t] == model.obj_dict[:x][r, t] * model.obj_dict[:ω][r, t])
                end
    
                let r = "WPCR"
                    if t == 1
                        @constraint(model, data.:需求[int2week(t)] == 75 + model.obj_dict[:x][r, t] - model.obj_dict[:y][r, t])
                    elseif t == 7
                        @constraint(model, data.:需求[int2week(t)] == model.obj_dict[:y][r, t-1] + model.obj_dict[:x][r, t] - model.obj_dict[:y][r, t])
    
                        @constraint(model, model.obj_dict[:y][r, t] == 0)
                    else
                        @constraint(model, data.:需求[int2week(t)] == model.obj_dict[:y][r, t-1] + model.obj_dict[:x][r, t] - model.obj_dict[:y][r, t])
                    end
                    @constraint(model, model.obj_dict[:x][r, t] == model.obj_dict[:x][r, t] * model.obj_dict[:ω][r, t])
                end
    
                @constraint(model, sum(model.obj_dict[:M][r, t] for r in element_key) <= data.:生产总工时限制[int2week(t)])
            end
        end
        return model, data
    end
    
    let data = let data = WPCR需求和关键设备工时限制(), (len, _) = size(data)
            let 生产总工时限制 = Dict((
                    let item = data[i, :]
                        String(item[:天]) => item["A、B、C生产总工时限制(工时)"]
                    end
                    for i in 1:len
                )),
                WPCR需求 = Dict(let item = data[i, :]
                    String(item[:天]) => item["WPCR需求(个)"]
                end
                              for i in 1:len),
                需求量 = Dict(
                    "A" => 3,
                    "B" => 4,
                    "C" => 5),
                组件需求量 = Dict(
                    "A" => Dict(
                        "A1" => 6,
                        "A2" => 8,
                        "A3" => 2),
                    "B" => Dict(
                        "B1" => 2,
                        "B2" => 4),
                    "C" => Dict(
                        "C1" => 8,
                        "C2" => 2,
                        "C3" => 12)),
                工时消耗 = Dict(
                    "A" => 3,
                    "B" => 5,
                    "C" => 5),
                product = product_Product(生产准备费用和单件库存费用(),
                    "WPCR",
                    需求量,
                    组件需求量,
                    工时消耗)
    
                Robot_data(
                    生产总工时限制,
                    WPCR需求,
                    product)
            end
        end,
        model = solve_robot(
            Gurobi.Optimizer,
            data,
            add_robot_variables,
            add_robot_constraints_2,
            add_robot_objective)
    
        CSV.write("第二题.csv", answer2df(model, data))
    end
    
    md"""
    ## 第三题
    """
    
    function third_write(model, path)
        third_answer(model) = model.obj_dict[:μ][:, 1] |> x -> zip(axes(x)[1], Vector(value.(x))) |> collect |> x -> filter(x -> x[2] == 1, x) |> x -> map(x -> x[1], x)
    
        title = append!(["第" * string(i) * "次" for i in 1:7], ["总成本\n"]) |> x -> join(x, ", ")
        values = append!(third_answer(model), [round(objective_value(model))]) |> x -> join(x, ", ")
        write(path, title * values)
    end
    
    function str2week(key_str)
        last = lastindex(key_str)
        first = prevind(key_str, last)
        return key_str[first:last]
    end
    
    function WPCR_30_to_Dict(data::DataFrame)
        WPCR_30_Dict = Dict()
        for (k1, v) in zip(1:100, eachcol(连续30周的WPCR需求数据()))
            int2week(int) = "第" * string(int - 1) * "周"
            int2week(int1, int2) = int2week(int1) * (["周一", "周二", "周三", "周四", "周五", "周六", "周日"][int2])
            if k1 != 1
                for (k2, v2) in zip(1:7, v)
                    WPCR_30_Dict[int2week(k1, k2)] = v2
                end
            end
        end
        return WPCR_30_Dict
    end
    
    function add_robot_variables_3(model, data)
    
        time = length(data.:需求)
    
        get_element(data) = data.:产品.:组件
        get_component(element) = element.:组件
    
        product = data.:产品
        element = get_element(data)
        components = [get_component(v) for (k, v) in element]
    
        let element_key = keys(element),
            component_key = keys.(components) |> x -> [[i for i in v] for v in x] |> x -> vcat(x...)
    
            @variable(model, x[["WPCR", element_key..., component_key...], [i for i in 1:time]] >= 0, Int)
            @variable(model, y[["WPCR", element_key..., component_key...], [i for i in 1:time]] >= 0, Int)
            @variable(model, ω[["WPCR", element_key..., component_key...], [i for i in 1:time]], Bin)
            @variable(model, M[[element_key...], [i for i in 1:time]] >= 0, Int)
            @variable(model, μ[[i for i in -4:time]], Bin)
            @constraint(model, [i = -4:0], model.obj_dict[:μ][i] == 0)
        end
        return model, data
    end
    
    function add_robot_constraints_3(model, data)
        time = length(data.:需求)
    
        week(day) = mod(day - 1, 7) + 1
        int2week(int) = ["周一", "周二", "周三", "周四", "周五", "周六", "周日"][week(int)]
        day2week(day) = "第" * string(div(day - 1, 7) + 1) * "周" * int2week(day)
        get_element(data) = data.:产品.:组件
        get_component(element) = element.:组件
    
        product = data.:产品
        element = get_element(data)
    
        let element_key = keys(element), r⁻ = "WPCR"
            for t in 1:time
                for r in element_key
                    let item = element[r], r⁻ = r, component_key = keys(item.:组件)
                        for r in component_key
                            if t == 1
                                @constraint(model, model.obj_dict[:y][r, t] == 0 + model.obj_dict[:x][r, t] - item.:需求量[r] * model.obj_dict[:x][r⁻, t])
    
                                @constraint(model, item.:需求量[r] * model.obj_dict[:x][r⁻, t] <= 0)
                            elseif t == time
                                @constraint(model, model.obj_dict[:y][r, t] == model.obj_dict[:y][r, t-1] + model.obj_dict[:x][r, t] - item.:需求量[r] * model.obj_dict[:x][r⁻, t])
    
                                @constraint(model, model.obj_dict[:y][r, t] == 0)
    
                                @constraint(model, item.:需求量[r] * model.obj_dict[:x][r⁻, t] <= model.obj_dict[:y][r, t-1])
                            else
                                @constraint(model, model.obj_dict[:y][r, t] == model.obj_dict[:y][r, t-1] + model.obj_dict[:x][r, t] - item.:需求量[r] * model.obj_dict[:x][r⁻, t])
    
                                @constraint(model, item.:需求量[r] * model.obj_dict[:x][r⁻, t] <= model.obj_dict[:y][r, t-1])
                            end
    
                            @constraint(model, model.obj_dict[:x][r, t] == model.obj_dict[:x][r, t] * model.obj_dict[:ω][r, t])
                        end
    
                    end
    
                    if t == 1
                        @constraint(model, model.obj_dict[:y][r, t] == 0 + model.obj_dict[:x][r, t] - product.:需求量[r] * model.obj_dict[:x][r⁻, t])
    
                        @constraint(model, product.:需求量[r] * model.obj_dict[:x][r⁻, t] <= 0)
                    elseif t == time
                        @constraint(model, model.obj_dict[:y][r, t] == model.obj_dict[:y][r, t-1] + model.obj_dict[:x][r, t] - product.:需求量[r] * model.obj_dict[:x][r⁻, t])
    
                        @constraint(model, model.obj_dict[:y][r, t] == 0)
    
                        @constraint(model, product.:需求量[r] * model.obj_dict[:x][r⁻, t] <= model.obj_dict[:y][r, t-1])
                    else
                        @constraint(model, model.obj_dict[:y][r, t] == model.obj_dict[:y][r, t-1] + model.obj_dict[:x][r, t] - product.:需求量[r] * model.obj_dict[:x][r⁻, t])
    
                        @constraint(model, product.:需求量[r] * model.obj_dict[:x][r⁻, t] <= model.obj_dict[:y][r, t-1])
                    end
    
                    @constraint(model, model.obj_dict[:x][r, t] * element[r].:工时消耗 == model.obj_dict[:M][r, t] * model.obj_dict[:ω][r, t])
    
                    @constraint(model, model.obj_dict[:x][r, t] == model.obj_dict[:x][r, t] * model.obj_dict[:ω][r, t])
                end
    
                let r = "WPCR"
                    if t == 1
                        @constraint(model, data.:需求[day2week(t)] == 75 + model.obj_dict[:x][r, t] - model.obj_dict[:y][r, t])
                    elseif t == time
                        @constraint(model, data.:需求[day2week(t)] == model.obj_dict[:y][r, t-1] + model.obj_dict[:x][r, t] - model.obj_dict[:y][r, t])
    
                        @constraint(model, model.obj_dict[:y][r, t] == 0)
                    else
                        @constraint(model, data.:需求[day2week(t)] == model.obj_dict[:y][r, t-1] + model.obj_dict[:x][r, t] - model.obj_dict[:y][r, t])
                    end
    
                    @constraint(model, model.obj_dict[:x][r, t] == model.obj_dict[:x][r, t] * model.obj_dict[:ω][r, t])
                end
    
                @constraint(model, sum(model.obj_dict[:M][r, t] for r in element_key) <= data.:生产总工时限制[int2week(t)] * (1 - model.obj_dict[:μ][t] + 0.1 * model.obj_dict[:μ][t-1] + 0.08 * model.obj_dict[:μ][t-2] + 0.06 * model.obj_dict[:μ][t-3] + 0.04 * model.obj_dict[:μ][t-4] + 0.02 * model.obj_dict[:μ][t-5]))
    
                @constraint(model, sum(model.obj_dict[:M][r, t] for r in element_key) <= data.:生产总工时限制[int2week(t)])
    
                if t >= 6
                    @constraint(model, sum(model.obj_dict[:μ][t] for t in t-5:t) <= 1)
                end
    
            end
    
            @constraint(model, sum(model.obj_dict[:μ][t] for t in 1:time) == 7)
    
        end
        return model, data
    end
    
    let data = let data = WPCR需求和关键设备工时限制(), (len, _) = size(data)
            let 生产总工时限制 = Dict((
                    let item = data[i, :]
                        String(item[:天]) => item["A、B、C生产总工时限制(工时)"]
                    end
                    for i in 1:len
                )),
                WPCR需求 = WPCR_30_to_Dict(连续30周的WPCR需求数据()),
                需求量 = Dict(
                    "A" => 3,
                    "B" => 4,
                    "C" => 5),
                组件需求量 = Dict(
                    "A" => Dict(
                        "A1" => 6,
                        "A2" => 8,
                        "A3" => 2),
                    "B" => Dict(
                        "B1" => 2,
                        "B2" => 4),
                    "C" => Dict(
                        "C1" => 8,
                        "C2" => 2,
                        "C3" => 12)),
                工时消耗 = Dict(
                    "A" => 3,
                    "B" => 5,
                    "C" => 5),
                product = product_Product(生产准备费用和单件库存费用(),
                    "WPCR",
                    需求量,
                    组件需求量,
                    工时消耗)
    
                Robot_data(
                    生产总工时限制,
                    WPCR需求,
                    product)
            end
        end
        model = solve_robot(
            Gurobi.Optimizer,
            data,
            add_robot_variables_3,
            add_robot_constraints_3,
            add_robot_objective,
            600)
        third_write(model, "第三题.csv")
    end
    
    md"""
    ## 第四题
    """
    
    第四题预测数据() = read_data("第四题 WPCR 需求和关键设备工时限制.csv")
    
    let data = let data = 第四题预测数据(), (len, _) = size(data)
            let 生产总工时限制 = Dict((
                    let item = data[i, :]
                        String(item[:天]) => item["A、B、C生产总工时限制(工时)"]
                    end
                    for i in 1:len
                )),
                WPCR需求 = Dict(let item = data[i, :]
                    String(item[:天]) => item["WPCR需求(个)"]
                end
                              for i in 1:len),
                需求量 = Dict(
                    "A" => 3,
                    "B" => 4,
                    "C" => 5),
                组件需求量 = Dict(
                    "A" => Dict(
                        "A1" => 6,
                        "A2" => 8,
                        "A3" => 2),
                    "B" => Dict(
                        "B1" => 2,
                        "B2" => 4),
                    "C" => Dict(
                        "C1" => 8,
                        "C2" => 2,
                        "C3" => 12)),
                工时消耗 = Dict(
                    "A" => 3,
                    "B" => 5,
                    "C" => 5),
                product = product_Product(生产准备费用和单件库存费用(),
                    "WPCR",
                    需求量,
                    组件需求量,
                    工时消耗)
    
                Robot_data(
                    生产总工时限制,
                    WPCR需求,
                    product)
            end
        end,
        model = solve_robot(
            Gurobi.Optimizer,
            data,
            add_robot_variables,
            add_robot_constraints,
            add_robot_objective)
    
        CSV.write("第四题.csv", answer2df(model, data))
    end
\end{lstlisting}
\end{appendices}