

 \maketitle
 %\tableofcontents
 \begin{abstract}

在我国史料中很早就有玻璃的相关记载,称作“璆琳琅玕”、“流离”等。
由于最早的制作的玻璃是是玉的仿制品且易受埋藏环境的影响而风化,而在丝绸之路引入了大量西方玻璃制品,导致我国古代玻璃制品鉴别难度较大。
要科学地考察中国古代玻璃的起源和发展,必须首先要经过科学鉴定,区分人工制造的玻璃、釉砂和玻砂,以及天然的玉石和宝石。
本文根据出土玻璃的外观与表面化学成分特征,建立了科学的回归分类模型,并为相关研究者提供参考。

对于问题一,本文对运用附件1中玻璃样品的类型、纹饰和颜色等特征数据对表面风化状态进行回归分析,建立了\textbf{Logistic线性回归}模型,并结合神经网络\textbf{反向传输}的特性,利用\textbf{梯度下降}算法计算出最优拟合参数。
首先对附件1中数据进行清洗并得到54条有效数据,再针对特征的特点进行标准化,得到样本特征矩阵。
在此基础上建立Logistic线性回归模型

\keywords{TOPSIS法\quad  熵权法\quad   LSTM模型\quad 线性规划}
\end{abstract}
