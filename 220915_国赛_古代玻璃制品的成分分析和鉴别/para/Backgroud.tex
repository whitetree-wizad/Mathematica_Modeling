
\section{问题重述}

玻璃的主要原料是石英砂,主要化学成分是二氧化硅(\ce{SiO2})。由于纯石英砂的熔点较高,为了降低熔化温度,在炼制时需要添加助熔剂。古代常用的助熔剂有草木灰、天然泡碱、硝石和铅矿石等,并添加石灰石作为稳定剂,石灰石煅烧以后转化为氧化钙(\ce{CaO})。添加的助熔剂不同,其主要化学成分也不同。

古代玻璃极易受埋藏环境的影响而风化。在风化过程中,内部元素与环境元素进行大量交换,导致其成分比例发生变化,从而影响对其类别的正确判断。

现有一批我国古代玻璃制品的相关数据,考古工作者依据这些文物样品的化学成分和其他检测手段已将其分为高钾玻璃和铅钡玻璃两种类型。

请你们团队依据附件中的相关数据进行分析建模,解决以下问题:

\begin{itemize}

\item \textbf{问题 1:}根据附件1,分析这些玻璃文物的表面风化与其玻璃类型、纹饰和颜色的关系。
再结合玻璃的类型,查阅附件2,分析有无风化时化学成分含量的统计规律,并根据风化点检测数据,预测其风化前的化学成分含量。

\item \textbf{问题 2:}首先分析高钾玻璃、铅钡玻璃的分类规律。
再根据合适的化学成分对每类玻璃进行亚类划分,分析结果的合理性和敏感性。

\item \textbf{问题 3:}分析附件3中未知类别玻璃文物的化学成分,并鉴别其所属类型,分析结果的敏感性。

\item \textbf{问题 4: }针对划分的不同类别玻璃文物样品,分析同类别中样品化学成分之间的关联关系,再比较不同类别之间关联关系的差异性。

\end{itemize}


\section{问题分析} % (fold)
\label{sec:问题分析}

\subsection{问题一} % (fold)
\label{sub:问题一}



% subsection 问题一 (end)




% section 问题分析 (end)

\section{符号说明} % (fold)
\label{sec:符号说明}

% section 符号说明 (end)