\section{数据预处理} % (fold)
\label{sec:数据预处理}

\subsection{附件1} % (fold)
\label{sub:附件1}

% subsection 附件1 (end)

\subsubsection{处理缺失值} % (fold)
\label{ssub:处理缺失值}

首先,附件1中提供的某些文物数据在颜色项存在缺失,这可能是由于文物出土条件较差,导致颜色已经无法辨别。
由于此缺失项无益于回归模型的建立,故去除所有包含缺失值的条目(文物编号为19、40、48、58)。

% subsubsection 处理缺失值 (end)

\subsubsection{变量二值化} % (fold)
\label{ssub:变量二值化}

为了便于数学描述,本文构建了二值变量$y^{(i)}$,$x^{(i)}_1$,$x^{(i)}_2$,$x^{(i)}_3$,以表示样品$i$的表面风化与否,类型和纹饰状态。

\begin{enumerate}
\item 风化状态
\begin{equation}
y^{(i)}=\left\{\begin{array}{l}
1, \quad \text { 文物 } i \text { 风化 }; \\
0, \quad \text { 文物 } i \text { 无风化 }.
\end{array}\right.
\end{equation}

\item 类型
\begin{equation}
	x_{1}^{(i)}=\left\{\begin{array}{l}
1, \quad \text { 文物 } i \text { 样品类型为高钾玻璃 }; \\
0, \quad \text { 文物 } i \text { 样品类型为铅钡玻璃 }.
\end{array}\right.
\end{equation}

\item 纹饰
\begin{equation}
	x_{2}^{(i)}=\left\{\begin{array}{l}
1,\quad \text { 文物 } i \text { 纹饰为A };  \\
0,\quad \text { 文物 } i \text { 纹饰不为A }.
\end{array}\right.
\end{equation}

\begin{equation}
	x_{3}^{(i)}=\left\{\begin{array}{l}
1,\quad \text { 文物 } i \text { 纹饰为B};  \\
0,\quad \text { 文物 } i \text { 纹饰不为B}.
\end{array}\right.
\end{equation}

文物$i$共有3三种纹饰状态,即A、B、C,若设置三个二值变量,则会导致完全多重共线性,从而落入虚拟变量陷阱\cite{伍德里奇2014计量经济学导论}。
故本文使用上述两个二值变量以表示以上三种状态,例如,当文物$i$的纹饰为C时,此时$x_{2}^{(i)}$、$x_{2}^{(i)}$即表示为0。
从附件1中44件文物样品的纹饰分布来看,纹饰为C的样品共有28件,较其他两种纹饰更多,因此可以将$x_{2}^{(i)}$、$x_{2}^{(i)}$的初始值均设为0,当纹饰为A、C时,则令其对应的二值变量取1。



\end{enumerate}



% subsubsection 变量虚拟化 (end)

\subsubsection{颜色RGB标准化} % (fold)
\label{ssub:颜色RGB标准化}

样品的颜色受到多种因素影响,呈现出的色彩各异。
同样地,若两个样品的化学组分和结构相似,表现出的色彩差异也较小(如深绿与浅绿)。
因此,直接将颜色当作独立的分类变量并不妥当,本文将色彩量化为RGB值以体现颜色之间的差异程度,并据此建立后文模型。

\begin{table}[!htp]
\caption{\bf RGB值对应表}
\setlength{\tabcolsep}{7mm}
\centering
\begin{tabular}{clccc}
\toprule
颜色~ &  & R   & G   & B    \\
\midrule
蓝绿  &  & 0   & 131 & 143  \\
深绿  &  & 0   & 100 & 0    \\
深蓝  &  & 18  & 18  & 110  \\
浅蓝  &  & 135 & 206 & 250  \\
浅绿  &  & 204 & 224 & 153  \\
紫   &  & 143 & 7   & 131  \\
绿   &  & 0   & 255 & 0    \\
黑   &  & 0   & 0   & 0    \\
\bottomrule
\end{tabular}
\end{table}

由于RGB数据量级与其他数据差距较大,因此对其进行正态标准化

\begin{equation}
	x_{j}^{(i)}=\frac{X_{j}^{(i)}-\mu_{j}}{\sigma_{j}}.
\end{equation}
其中,$X_j^{(i)}(j=4,5,6)$分别对应文物$i$颜色RGB值的三个分量,$\mu_j$及$\sigma_j$为对应分量的期望和标准差。

\begin{equation}
	\begin{array}{c}
\mu_{j}=\frac{1}{54} \sum_{i=1}^{54} X_{j}^{(i)};\\
\sigma_{j}=\sqrt{\frac{1}{53} \sum_{i=1}^{54}\left[X_{j}^{(i)}-\mu_{j}\right]^{2}}; \\
\left(\begin{array}{ccc} x_{4}^{(i)}& x_{5}^{(i)}& x_{6}^{(i)}\end{array}\right)=\left(\begin{array}{ccc}\frac{R^{(i)}-\mu_{4}}{\sigma_{4}}& \frac{G^{(i)}-\mu_{5}}{\sigma_{5}} &\frac{B^{(i)}-\mu_{6}}{\sigma_{6}}\end{array}\right)
\end{array}
\end{equation}

% subsubsection 颜色rgb标准化 (end)

% section 数据预处理 (end)