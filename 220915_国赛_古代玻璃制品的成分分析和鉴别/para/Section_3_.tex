\section{化学成分含量}

文物$i$采样点的14种化学成分含量为
\[ \boldsymbol{X^{(i)}} = \left(\begin{array}{cccc}
     x_7^{(i)} & x_8^{(i)} & \ldots & x_{20}^{(i)}
   \end{array}\right) \]
结合玻璃的类型,将高钾玻璃、铅钡玻璃分开分析,每种玻璃类型中又根据有无风化分为两类。

\section{方差齐性检验}

\subsection{为什么要做}

必要性:

1.后文进行的多因素方差分析首先假设样品呈正态分布,因此需要通过方差齐性检验

2.后文分析文物表面有无风化化学成分含量的统计规律,需要保证风化前后样本化学成分含量在整体中分位数不变(本质上是确保了概率密度函数形状相同)

\subsection{为什么能做}

充分性:

由于风化对于不同样品作用效果类似,故风化前后文物的化学成分的相对含量得以保持

\subsection{$\mu_1 、 \mu_2 未知,显著性水平为
\alpha$(引用盛骤p194-195)}

原假设$H_0 : \sigma^2_1 = \sigma^2_2$,备择假设$H_1 : \sigma^2_1 \neq
\sigma^2_2$,

检验统计量
\[ F = \frac{S^2_1}{S^2_2} \]
拒绝域
\[ F \geqslant F_{\alpha / 2} (n_1 - 1, n_2 - 1) 或F \leqslant F_{1 - \alpha
   / 2} (n_1 - 1, n_2 - 1) \]

\section{多因素方差分析(引用陈希儒p321-324)}

Step 1 首先算出全部试验值的算数平均及总平方和$\mathit{SS}$

Step 2
估计各因素各水平的效应等于该因素水平的所有观察值的算数平均减去全部观察值的算数平均

Step 3 计算各因素的平方和、误差平方和及其自由度

Step 4 列出方差分析表

Step 5 查$F分布表$,得因素效应显著性

\section{文物表面有无风化化学成分含量映射(引用数分p10)}

建立映射
\begin{eqnarray*}
  h : & I \rightarrow K & \\
  & x_j^{(i)} \mapsto x_j^{(k)} &
\end{eqnarray*}
其中,$x_j^{(i)}$为文物$i$风化后化学成分$j$的含量,且服从正态分布:
$x_j^{(i)} \sim N (\mu_1, \sigma^2_1) ,分布函数为 \Phi_1$;
$x_j^{(k)}$为文物$i$风化前化学成分$j$的含量,
且服从正态分布: $x_j^{(k)} \sim N (\mu_2, \sigma^2_2)
,分布函数为 \Phi_2$。映射$h$满足
\[ \Phi_1 (x_j^{(i)}) = \Phi_2 (x_j^{(k)}) \]
解得
\[ x_j^{(k)} = \mu_2 + \frac{\sigma^2_2}{\sigma^2_1} [x_j^{(i)} - \mu_1] \]

\section{问题1.3} % (fold)
\label{sec:问题1_3}

% section 问题1_3 (end)

根据风化点检测数据,进一步将风化后的文物分为3类,分别为:

浅风化—文物表面风化,采样点未风化;

风化—文物表面及采样点均风化;

严重风化—文物表面风化,采样点严重风化。

进而采用上述映射,预测得到风化前的化学成分含量,详见附录。这不仅得到了预测结果,也证明了上述映射关系的确成立、真实可靠
