
原材料是企业重要的经营性资产,是企业维持正常生产经营活动的基础,原材料管理的好坏直接关系到企业的营运资金利用效率,也关系到企业生产经营的持续性\cite{杨庆军2012浅谈}。
改良生产企业的原材料管理方案,能提高原材料的周转率,有效降低仓储成本,也有利于保障生产流程的正常运行,在较大程度上加强企业竞争力\cite{李红燕2018论企业原材料的管理和研究}。
近年来,随着经济的稳步发展,企业在面对新冠疫情冲击的同时,还面临着更大规模、更高强度的竞争,优化原材料管理方案也较以往更具现实意义。%需修改

\section{问题分析}
制造商为了达到每日生产指标,以满足市场需求,需要向供货商(提供企业所需的原材料)提交订单。
由于运输、交付等问题,订购的原材料不能直接用于生产,而需先交付物流转运至仓库,以便企业生产时取用。

现实中,供应商在接到订单后,常因为生产能上限和供应优先度等问题调整原材料产量,导致实际按时提供的原料量与生产需要存在出入。
而此时制造商已经完成原料采购,不能通过更改之前的订购方案来弥补缺少原料,只能使用已经转移至仓库中的原料进行加工。
若原材料库存不足,原材料的缺口会导致生产线停摆,造成严重的经济损失。
为了减轻原料供应量波动对生产的影响,生产企业可提前储存一些原料,减轻生产停摆带来的损失。
但与此同时,过多的原材料积存会显著增加企业的仓储成本。
因此,为了改良生产企业的原材料管理方案,可根据以往交易数据,推测该供应商供应量的波动幅度,以制定出更加灵活的订购方案,削弱仓储成本,为企业带来更大收益。

供应商提供A、B、C三种原材料,容易发现,A、B、C三种类型的原材料在采购单价不相等的同时,生产单位商品的消耗量也是不同的。
其中,A类型和C类型原材料生产单位商品的采购费用相等,且略低于B类型原材料(由于相差小于1\%,本文暂且忽略),进而认为三者的本征采购单价相等。
为了便于研究,本文先对原材料做了等价代换:将其供应量以原材料的商品生产能力表示,采购单价也以本征采购单价表示。
%需修改
为了最大化经济效益,帮助生产企业做出决策,本文建立了供应量可预测的原材料订购、库存方案优化模型。

\subsection{问题一}
题目中要求对附件一中402家供应商进行量化分析,并建立供应商选择的评价指导体系,确定44家最重要的供应商。
供应商的重要程度受多种因素影响,通过查阅相关文献,利用定量分析的方法,我们选取出适当的指标并根据熵值赋予权重,建立基于熵权逼近理想解方法的供应商排序模型,最后对供应商进行排序优选。


\subsection{问题二}
题目中要求在问题一结果的基础上,选出能满足企业生产需求的最少供应商,针对这些供应商分别制定最经济的订购方案和损耗最小的转运方案。
我们首先利用0-1规划选出必要的供应商,同时为了保证原材料供应量充足,我们运用LSTM模型预测出转运商的供应水平,并建立了原材料订购方案的线性规划模型。
通过对供应过程的计算机仿真,分析订购和转运模型的实施效果。

\subsection{问题三}
题目中要求在尽量多地采购A类和尽量少地采购C类原材料的前提下,制定最经济的原材料订购方案和损耗最小的转运方案,并分析方案的实施效果。
本文发现,对原材料种类的限制,能通过对问题二模型进行扩展,在订购量分配时体现企业对原材料的偏好,进而通过优化模型制定了订购和转运方案。
模型求解后,仍经过计算机仿真,分析模型的优化效果。

\subsection{问题四}
题目中要求在供应商和转运商有限的情况下,评估企业每周产能的上限。
本题取消生产企业每周产能上限的同时,也消除了原材料的库存问题。
而通过分析供货数据不难发现,向企业输送的最大供给量主要取决于转运商的转运能力。
本问题也转化为在转运商以最大能力运输原材料时,一家供应商每周供应的原材料尽量由一家转运商运输。
因此,本文主要对转运商的转运方案进行优化,并据此给出未来24周的订购和转运方案。

最后我们对模型进行了中肯的评价和适当的推广。
