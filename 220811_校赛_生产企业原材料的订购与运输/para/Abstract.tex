

 \maketitle
 %\tableofcontents
 \begin{abstract}

近年,随着世界全球化加深,传统的供应关系已经不再适应于当今生产需要,其中原材料供应是企业供应链的主要环节。
本文提出了原材料供应数学模型的基础假设,给出企业对供应过程中订购与运输方案的建议。

对于问题一,我们通过对供货商供货特征的定量分析,取得描述供应商重要程度的评价指标,建立了基于熵权逼近理想解方法的供应商排序模型。
其中,我们主要从供应商的产业规模、供货能力、违约率、波动情况,来衡量此供应商对该企业正常生产的重要性。
为了定量并客观地评价每个供应商的重要程度,我们在模型中使用了熵权,以有效避免多因素评价模型当中确定权重时的主观性。
通过比较对象与理想解和负理想解的距离,得到出对象与理想解的相似度,最后排序优选确定最重要的50家供应商。

对于问题二,我们建立线性规划模型以对订购方案进行优化,并制定出未来24周原材料最经济的订购方案。
其中,在供应商的二次优选中,我们注意到每个供应商仅有选择与否两种状态,故采用0$\textendash$1规划取得满足供应需求的最少的44家供应商;考虑到原材料储备对保障企业生产的重要意义,制定订购方案前,我们首先应用LSTM模型预测选定的供应商的供应水平,确保企业供应链稳定与生产安全,进一步优化订购方案。
根据得到的订购方案制定损耗最小的转运方案,最后通过计算机程序仿真,以模拟优化后方案的实施效果。

对于问题三,由于企业为压缩生产成本,尽量多地采购 A 类和尽量少地采购 C 类原材料。
我们对问题二中线性规划模型进行扩展,减少生产企业对C类原材料供应商的依赖,并将A类原材料的采购优先度提高,构建线性规划模型,解出最优采购方案和转运方案。
最后由计算机仿真体现方案的实施效果。

对于问题四,我们不再考虑因企业生产技术有限产生的产量上限,而通过优化订购方案、转运方案最大化供应链的供应能力。
而通过分析供货数据我们发现,每周转运商的总转运能力远少于供给商的供给能力。
因此,我们以对转运商转运方案的优化为主,通过对模型的不断反思与改进,得出了未来24周的订购和转运方案。

最后我们对模型进行了中肯的评价和适当的推广。
\keywords{TOPSIS法\quad  熵权法\quad   LSTM模型\quad 线性规划}
\end{abstract}
