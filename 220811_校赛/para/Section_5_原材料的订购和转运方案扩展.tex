
\section{原材料的订购和转运方案扩展}

本题要求在尽量多地采购A类和尽量少地采购C类原材料的前提下,制定最经济的原材料订购方案和损耗最小的转运方案,并分析方案的实施效果。

\subsection{制定订购方案}

该企产成本,要减少转运及仓储的成本。
三类原材料运输和业为了压缩生存储的单位费用相同,但A类原材料的商品生产能力比C类原材料强,因此企业会尽量多地采购A类和尽量少地采购C类原材料。
本文忽略了原材料类型偏好对每周订货总量的影响,而主要关注于供应商订购量分配的变化。
因此,问题2中的每周订货总量计划保持不变。

问题2中,原材料的订购数量是直接根据供应商的评价来进行分配的。
问题1中建立的评价模型并未重点关注供应商贩卖的原材料类型,本文通过依据供应商贩卖的原材料类型,对供应商的评价加分,以尽量多地采购A类和尽量少地采购C类原材料。
考虑到综合评价指标均小于1分,本文对售卖A类原材料的供应商加1分,令企业尽量多地采购A类原材料。
联系上文,B类原材料的本征采购单价实际上略高,因此,本文没有对C类原材料的供应商扣分。

\subsection{制定运输方案}

因为A类原材料的采购单价较高,因此对转运损耗更加敏感。
问题3中企业计划尽量多地采购A类原材料,所以迫切需要制定合理的运输方案,以确保转运商的转运损耗率尽量少,来达到压缩生产成本的目的。
问题2中侧重于经济效益,要求损耗最少,而问题3要求转运损耗率尽量少,因此对问题2中的0$\textendash$1规划模型稍加改动:



式中$L'_i$代表第$i$周的转运损耗量,由于希望转运损耗率尽量少,故此处转运损耗量应取最小值。