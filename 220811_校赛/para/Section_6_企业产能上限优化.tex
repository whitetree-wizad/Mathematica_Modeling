
\section{企业产能上限优化}

题目中要求在供应商和转运商有限的情况下,评估企业每周产能的上限。
本题在取消生产企业每周产能上限的同时,也消除了原材料的库存问题,这使得供应商和转运商的供应量可直接由生产企业转化为产能。
本问题也转化为在供应商和转运商有限的情况下,求供应链可向企业输送的最大供给量。

联系附件数据,我们得知近5 年来,共有402 家供应商通过8家转运商向企业供货。
而通过分析供货数据我们发现,每周转运商的总转运能力远少于供给商的供给能力,故向企业输送的最大供给量主要取决于转运商的转运能力。
因此,本模型主要对转运商的转运能力进行优化。

本文引申一家供应商每周供应的原材料尽量由一家转运商运输,提出运筹目标为参与转运的转运商次最少:
\begin{equation}
\begin{array}{l}
\min \sum_{j=1}^{402} \sum_{l=1}^{8}\left[\frac{\theta_{i j l}}{\psi_{i j}}\right] \\
\text { s.t. }\left\{\begin{array}{l}
\sum_{i=1}^{8} \theta_{i j l}=\psi_{i j} \\
\sum_{j=1}^{402} \theta_{i j l}=6000
\end{array}\right.
\end{array}
\end{equation}
式中,$[]$是高斯函数,代表向下取整;$\theta_{i j l}$是第$i$周转运商T$l$为供应商S$j$运输的供货量。
为保证企业产能充分发挥,约束了每家转运商均取得其最大转运能力6000立方米。

事实上,由于A类原材料生产商品的能力能力最强,企业此时仍对其表现出明显的倾向。