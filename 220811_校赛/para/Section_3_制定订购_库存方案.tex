
\section{制定订购、库存方案}
\subsection{基于订货量的供货量确定}

在生产厂家选择确定的供应商并发出订单后,供应商在其供货能力范围内将尽可能按照订货量进行供货,实际供货量受其自身供货能力的影响;为了模拟现实中供货量的波动情况,本文令实际供货量偏差符合该供货商过去的供货偏差分布。

预测的供货量为:
\begin{equation}
    u_{i j}=\left\{\begin{array}{l}
E_{j} v_{i j}, \quad v_{i j} \leq \psi_{i j}, \\
E_{j} \psi_{i j}, \quad v_{i j} \geq {\psi}_{i j};
\end{array}\right.
\end{equation}

\noindent 其中,$\delta_n$为第$n$家供应商的绝对偏差的期望值。


\subsection{制定订购方案}

为获得最高收益,在制定订购方案时,需要平衡原料库存量。在供应充足时适当扩大库存,以填补未来原材料不足造成的的缺口,使其生产活动稳定进行。

分析预测结果,虽然平均每周的供货总量充足(大于$2.84 \times 10^4 m^3$),但仍有部分周供应不足,记缺货量为$D_{\omega}=2.84 \times10^6-z_{\omega,n}$(供货量充足时 ,$D_{\omega}=0$)。
为了在供应不足时也能尽可能地发挥产能,该企业会在供货总量充足的时候中增加进货,即额外订购量$d_{\omega}=z_{\omega,n}-2.84 \times 10^6$(供货总量低于$2.84 \times 10^4 m^3$时,$d_{\omega}=0$)。
% ???
为了保证订购的原材料均被加工为产品,要求$\sum_{\omega=1}^{24}D_{\omega} \geq \sum_{\omega=1}^{24}d_{\omega}$

令该企业收益最大,构建线性规划模型:

\begin{equation}
\begin{split}
{\bf max}\quad &p_1U-p_2u_{\omega} \\
{\bf s.t.}\quad &\sum_{\omega=1}^{24}D_{\omega} \geq \sum_{\omega=1}^{24}d_{\omega}
\end{split}
\end{equation}

\noindent 其中,$U$为24周内的总产量,$U=24 \times 2.84\times10^6-\sum_{\omega=1}^{24}D_{\omega}+\sum_{\omega=1}^{24}d_{\omega}$;
每周的原材料库存量$u_{\omega}=2.84-\sum_{\omega=1}^{\omega}D_{\omega}+\sum_{\omega=1}^{\omega}d_{\omega}$。
$p_1U$为相对库存费用的毛利润,$p_1$为单位产品的毛利润率;$p_2$为单位产品所需原料的平均储存成本。

经过以上的优化,本文已经确定了每周的订货总量,需根据各供货商供货能力分配订单。本文依据\ref{选优}中的各供应商选优顺序,向筛选后的供应商依次订货,直到预计供货量达到预期。