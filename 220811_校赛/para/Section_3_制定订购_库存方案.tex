
\section{制定订购、库存方案}
\subsection{基于订货量的供货量确定}

在生产厂家选择确定的供应商并发出订单后,供应商在其供货能力范围内将尽可能按照订货量进行供货,实际供货量受其自身供货能力的影响;为了模拟现实中供货量的波动情况,本文令实际供货量偏差符合该供货商过去的供货偏差情况。

预测的供货量为:
\begin{equation}
    u_{i j}=\left\{\begin{array}{l}
E_{j} v_{i j}, \quad v_{i j} \leq \psi_{i j}, \\
E_{j} \psi_{i j}, \quad v_{i j} \geq {\psi}_{i j};
\end{array}\right.
\end{equation}

当订货量高于供应商的能力时,供货量不再随订货量增长,可令此时订货量等于供应能力。
在此基础上,本文在后文中通过约束$v_{i j} \leq \psi_{i j}$,解决了分段函数线性化的问题。


\subsection{制定订购方案}

为获得最高收益,在制定订购方案时,需要平衡原料库存量。在供应充足时适当扩大库存,以填补未来原材料不足造成的的缺口,使其生产活动稳定进行。
为了在尽可能保证产量的前提下,使存储成本最小,本文建立了如下线性规划模型以确定每周的订货量。

分析预测结果,虽然平均每周的供货总量充足(大于$2.84 \times 10^4 m^3$),但仍有部分周供应不足,记缺货量为$D_i=2.84 \times10^6-u_i$(供货量充足时 ,$D_i=0$)。
为了在供应不足时也能尽可能地发挥产能,该企业会在供货总量充足的时候中增加进货,即额外订购量$d_i=u_i-2.84 \times 10^6$(供货总量低于$2.84 \times 10^4 m^3$时,$d_i=0$)。

为了保证订购的原材料均被加工为产品,要求$\sum_{i=1}^{24}D_i \geq \sum_{i=1}^{24}d_i$。
% 此时总可通过调整周期起点的方式,使得在周期结束时订购的原材料均被加工为产品。


令目标函数为企业利润,使企业收益最大,构建线性规划模型:

\begin{equation}
\begin{split}
{\bf max}\quad &p_1\eta-p_2\mu_i \\
{\bf s.t.}\quad &\sum_{i=1}^{24}D_i \geq \sum_{i=1}^{24}d_i
\end{split}
\end{equation}

\noindent 其中,$\eta$为24周内的总产量,$\eta=24 \times 2.84\times10^6-\sum_{i=1}^{24}D_i+\sum_{i=1}^{24}d_i$;
第$t$周的原材料库存量$\mu_t=2.84-\sum_{i=1}^{t}D_i+\sum_{i=1}^{t}d_i$。
$p_1$为单位产品的售价,则$p_1\eta$为企业24周的生产所得;$p_2$为单位产品所需原料的平均运输和储存成本。
该模型中的$p_1$和$p_2$是关键参数:当$p_1$较大时,企业会倾向多投入存储费用来保证产能不被浪费,以生产更多产品赚取高额利润;当$p_2$较大时,企业则会牺牲产量来避免支付昂贵的仓储费用。
本文在测试时,产品的售价约比仓储成本高一个数量级。
需要泛用时,$p_1$和$p_2$应根据实际情况灵活取值。

以上模型的决策变量是订购量$v_{ij}$,得到的最优解代表着每周最经济的订货量。
此处由于三种类型原材料的本征采购单价相等,因此在进行运筹时并没有在目标函数中计算。
另外,通过对结果的检验,本文发现最优解得订购方案在转运商的能力范围之内,因此忽略了转运商的能力对订货量的限制。

经过以上的优化,本文已经确定了每周的订货总量,需根据各供货商供货能力分配订单。本文依据\ref{选优}中的各供应商选优顺序,向筛选后的供应商依次订货,直到预计供货量达到预期。

\subsection{制定转运方案}

为了保证生产的经济效应,企业不仅需要制定订货方案,计划转运商的运输方案也是必要的,本文希望根据上文制定的订货方案进一步计划损耗最少的转运方案。
在实际转运过程中,原材料会有一定的损耗,损耗量占供货量的百分比称为“损耗率”。
通常情况下,一家供应商每周供应的原材料尽量由一家转运商运输。
为体现供应商S$j$选择的转运商T$l$,引入0—1变量
% \begin{equation}
%     \rho_{ijl}}=\left\{\begin{array}{l}
% 0, , \\
% 1}, ;
% \end{array}\right.
% \end{equation}
假设同一家转运商T$l$的损耗率总是恒定的$\iota_l$,则有:
\begin{equation}
\begin{split}
{\bf max}\quad &L_i=\sum_{l=1}^{8}(\iota_l\sum_{j=1}^{44}\rho_{ijl}\psi_{ij}p_3 \\
{\bf s.t.}\quad &\sum_{l=1}^{8}\rho_{ijl} =1
\end{split}
\end{equation}
式中$p_3$是原材料的采购单价,根据供应商贩卖的原材料类型确定。
目标函数$L_i$是第$i$周的损失价值,故希望最小。
该0—1规划模型每周运筹一次,制定该周的转运方案。
约束条件表示一家供应商每周供应的原材料只由一家转运商运输,且每家转运商的运输原料的能力为6000立方米/周。
事实上,根据预测模型,发现有的供应商单周供应量大于6000立方米,此时便无法只由一家转运商运输了。
但这种情况发生次数较少,通过手工分配后,转运效果整体良好。
